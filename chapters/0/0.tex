\ifdefined\latexbokFontdir\else\def\latexbokFontdir{../../fonts}\fi
\ifdefined\latexbokFiguredir\else\def\latexbokFiguredir{../../examples}\fi
\documentclass[lang=sv,ptsize=10pt,font=none,nomath,titles=bf,../../a4.tex]{subfiles}
\begin{document}
\nntocsection{Inledning}
Att publicera något, oavsett om det är långa böcker eller korta artiklar,
är inte helt enkelt. Förr, innan det fanns en dator i varje hem och innan
programvaror som Microsoft Word och Open\-Office blev tillgängliga för var
man så var publicering något komplext, något som utfördes av många olika
personer. Arbetet delades upp i olika bitar och en expert inom varje
område fick sköta den biten; författande, typsättning, design och så
vidare.

Moderna programvaror (ordbehandlare, främst) fungerar på ett helt annat
sätt. De ger all makt till författaren, som kanske varken vill eller bör
(författaren är ju inte typsättare) ha makt över typografin. Som en konsekvens
blir typsättningen av dokument inte alltid bra — användarna är under
illusionen att ett estetiskt tilltalande dokument även är ett bra dokument
rent typografiskt. Så är givetvis inte fallet.

\index{separation!innehåll och stil}
\index{semantik}
För att få snygga dokument måste man återskapa den gamla ordningen där
varje uppgift löses av någon som är bra på det. En författare ska inte
behöva berätta \emph{hur} saker ska se ut — det ska designern göra —
författaren ska bara berätta \emph{vad} saker är\footnote{Detta är inte
helt olikt den uppdelning som görs mellan HTML och CSS.}. \LaTeX{} låter
dig som författare göra just detta.

\phantomsection
\addcontentsline{toc}{subsection}{Vad är \LaTeX{}?}
\subsection*{Vad är \LaTeX{}?}
\LaTeX{} (uttalas \textipa{\tipafont[la\textprimstress tES]}) är i princip en
påbyggnad till \TeX{}, ett typsättningssystem. Man kan se det som att
\LaTeX{} är designern, och \TeX{} är typsättaren. \LaTeX{} berättar i
någon mening för \TeX{} hur den tror att du vill ha ditt dokument (utifrån
dokumentklassen samt den kod du skriver) och låter sedan \TeX{} typsätta
detta för att skapa ett ”tryckt” dokument.

\index{kompilator}
\index{kompilator!pdfLaTeX@kompilator!\pdfLaTeX}
\index{typsättningssystem}
Men \LaTeX{} är inte bara ett typsättningssystem; både språket dokumenten
skrivs i och den kompilator som omvandlar dokumenten till \DVI-filer
kallas \LaTeX{} (men just kompilatorn brukar benämnas \cli{latex},
eftersom det är så den körs från terminalen). Det finns även modernare
versioner av \LaTeX{}-kompilatorn, till exempel \pdfLaTeX{} som skapar
\PDF-dokument
 direkt (så att man slipper konvertera dem med \cli{dvipdf}) och
\XeTeX{}, som baseras helt på \textsc{Unicode} och har bättre stöd för
diverse typsnitt.

\phantomsection
\addcontentsline{toc}{subsection}{Varför \LaTeX{}?}
\subsection*{Varför \LaTeX?}
Fördelen med \LaTeX{} är att författaren endast behöver lära sig några
enkla kommandon — sådana som gör texten kursiv eller infogar en figur
— men att dokumentet ändå håller mycket hög standard, precis som om
det typsatts av en riktig typsättare.

\index{stavningskontroll}
Eftersom språket man skriver dokument i uppmanar till struktur och ordning
blir det dessutom lättare att skriva strukturerade texter. Även om saker
som finns i en vanlig ordbehandlare (stavningskontroll, till exempel) saknas i
själva språket så kan (och bör) dessa tillhandahållas av externa program,
vilket gör att \LaTeX{} kan koncentreras på att vara bra på \emph{en} sak:
att typsätta dokument. Saker som är komplicerade eller omöjliga att göra
i en vanlig ordbehandlare, till exempel korsreferenser, figurer, tabeller,
fotnoter, referenslistor och dylikt är mycket enkla att göra i \LaTeX{}
och oftast fullt automatiserade.

Dessutom har \LaTeX{} traditionellt använts av akademiker eftersom det gör 
typsättning av just matematik och fysik både enkelt och visuellt attraktivt. 

\phantomsection\vspace{-1em}
\addcontentsline{toc}{subsection}{I denna introduktion}
\subsection*{I denna introduktion}
Den här korta introduktionen kommer att visa dig hur man på ett enkelt
sätt typsätter dokument med \LaTeX{} i vanliga tillämpningar.
Dessutom kommer den framåt slutet peka på specifika paket eller
resurser som kan vara användbara för mer avancerade fall.

Efter att ha läst den här introduktionen bör läsaren kunna skriva
dokument och rapporter utan problem. Det är dock inte tänkt att denna
introduktion ska vara en fullgod referens till \LaTeX; för detta
rekommenderas istället \textcite{Lamport94} och
\textcite{Mittelbach04}.

Introduktionen innehåller följande delar:
\begin{description}
	\item[{\Cref{sec:1}, \hyperref[sec:1]{Grundläggande begrepp}}]
	beskriver den grundläggande strukturen hos \LaTeX-dokument och hur det
	språk dokumenten skrivs i fungerar i korta drag. Efter denna del bör
	du veta på ett ungefär hur \LaTeX{} fungerar.
	
	\item[{\Cref{sec:2}, \hyperref[sec:2]{Typsättning med \pdfLaTeX}}]
	beskriver i detalj hur man skriver ett vanligt
	\LaTeX-dokument, och förklarar några av de viktigaste miljöerna
	och kommandona som används. Efter denna del bör du kunna skriva enkla
	dokument med \LaTeX.
	
	\item[{\Cref{sec:3}, \hyperref[sec:3]{Matematik med \LaTeX{} och 
	\AmS}}]
	beskriver hur man på bästa sätt använder \LaTeX{} tillsammans med
	\AmS-paketen för att typsätta det \LaTeX{} typsätter bäst, matematik,
	och går även kort in på hur man typsätter en del fysik med paketet
	\pack*{siunitx}.
	
	\item[{\Cref{sec:4}, \hyperref[sec:4]{Grafik med \LaTeX}}]
	beskriver hur man inkluderar grafik i \LaTeX{} med paketet
	\pack*{graphicx}, och visar några korta exempel på hur man kan rita
	direkt i \LaTeX{} med \PGFTikZ{}. Efter den här och föregående del bör
	du kunna skriva fullständiga rapporter med \LaTeX.
	
	\item[{\Cref{sec:5}, \hyperref[sec:5]{Referenser med \BibTeX}}]
	beskriver hur du använder \BibTeX{} för att hålla koll på och använda
	referenser i \LaTeX, och beskriver även i korthet paketet \pack*{chscite} som
	hjälper dig att typsätta referenser på det sätt Chalmers bibliotek
	rekommenderar. Efter denna delen bör du kunna skriva långa arbeten
	(till exempel kandidatrapporter) i \LaTeX.
	
	\item[{\Cref{sec:6}, \hyperref[sec:6]{Vidare läsning}}]
	tipsar om andra resurser, paket, dokumentklasser och rekommendationer
	som kan vara av nytta när du skriver långa (eller korta) rapporter.
	Kan vara en språngbräda om du vill göra något som inte förklaras i
	resten av introduktionen.
	
	\item[\Cref{app:1}, {\hyperref[app:1]{En enkel mall}}]
	innehåller en mall du med fördel kan basera dina framtida 
	\LaTeX{}-dokument
	på. Den är fullt kommenterad och motiverar de paket som inkluderas och
	kommandon som definieras.
\end{description}

Det är viktigt att läsa delarna i rätt ordning; varje del bygger på de
föregående, och de är ju trots allt inte särskilt långa. Se till att
studera och förstå de exempel som presenteras, och lek gärna lite själv
om du inte riktigt förstår. Det finns inget bättre sätt att lära sig än
att vara nyfiken!

\index{installera!LaTeX@installera!\LaTeX{}}
\LaTeX{} finns till många plattformar, och finns installerat på Chalmers
Linux\-da\-to\-rer. Vill du installera \LaTeX{} på din egen dator finns det
sannolikt i din pakethanterare (om du använder Linux), alternativt i form
av \TeX{} Live\footnote{\url{http://www.tug.org/texlive/}}. Använder du
Mac OS~X finns det istället
Mac\TeX\footnote{\url{http://www.tug.org/mactex/}}, och till Windows finns
MiK\TeX\footnote{\url{http://miktex.org/}}. Den här introduktionen kan
tyvärr inte ge fullständiga instruktioner för att installera dessa paket
(det är inte introduktionens syfte);
konsultera istället respektive pakets dokumentation.

\index{Comprehensive \TeX{} Archive Network, The}
Under introduktionens gång kommer det refereras till så kallade
\emph{paket}, som används för att utöka \LaTeX{} med intressanta (och
ibland nödvändiga) funktioner. Dessa paket kommer oftast att beskrivas
lite kort, men vill man se fullständig dokumentation för varje paket
kan man leta på \emph{the Comprehensive \TeX{} Archive Network}
(CTAN)\footnote{\url{http://www.ctan.org/}\label{sec:ctan}}.
Det lättaste sättet att hitta paket på CTAN är att använda dess
sökfunktion\footnote{\url{http://www.ctan.org/search/}}.
\end{document}

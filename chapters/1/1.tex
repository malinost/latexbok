\ifdefined\latexbokFontdir\else\def\latexbokFontdir{../../fonts}\fi
\ifdefined\latexbokFiguredir\else\def\latexbokFiguredir{../../examples}\fi
\documentclass[lang=sv,ptsize=10pt,font=none,nomath,titles=bf,../../a4.tex]{subfiles}
\begin{document}
\part{Grundläggande begrepp}\label{sec:1}
\chapter{\LaTeX-dokument}
Introduktionen har förklarat lite kort vad \TeX/\LaTeX{} är och varför du
bör använda systemet för att typsätta dina rapporter, artiklar och böcker.
Denna del kommer att inleda din \TeX-bana genom att lite kort förklara hur
ett \LaTeX-dokument är uppbyggt, några viktiga begrepp samt hur
\LaTeX-kompilatorn (i det här fallet \cli{pdflatex}) fungerar.

De dokument \LaTeX{} läser in är enkla textfiler, oftast med filändelsen
\cli{.tex}. Dessa kan skapas med vilken textredigerare som helst,
men det rekommenderas att man använder
en textredigerare med syntaxfärgning (då det underlättar felsökande) och
stavningskontroll. Oftast sparar man filen med teckenkodningen \UTF{} 
(detta gör de flesta moderna textredigerare), men det finns en risk att
filerna sparas med teckenkodningen \textsc{iso-8559-1}\footnote{Det finns 
även andra teckenkodningar som kan dyka upp{;} Windows-1252 är en sådan.}, 
och då måste man ha koll på detta eftersom man måste berätta för \LaTeX{}
hur filen ska läsas in.

\section{Tomrum}
Tomrumstecken, det vill säga mellanslag, tabbar och liknande behandlas
alla som ”tomrum” av \LaTeX{}. Flera sådana tecken i rad tolkas som ett
enda tomrum, vilket gör att man kan ex. indentera textstycken i sin
\LaTeX-fil utan att detta påverkar resultatet. Tomrum i början av en rad
ignoreras generellt sett, och ett enkelt nyradstecken tolkas också som
tomrum.

Två nyradstecken på rad (det vill säga en tom rad mellan två textrader) tolkas som
styckesindelning, och flera tomma rader efter varandra tolkas som en enda
tom rad. Detta gör att \LaTeX{}-filerna blir mycket enkla att både skriva
och läsa, även om man inte är en särskilt bevandrad \TeX{}niker.

\section{Specialtecken}
Vissa tecken är speciella för \LaTeX{}. Dessa används internt av \LaTeX{} 
för att representera speciella konstruktioner, och kan inte användas hur som helst i ett \LaTeX-dokument. Gör man det ändå
slutar det vanligtvis bara med att tecknet inte syns, men det kan
potentiellt göra att ditt dokument inte ens kompilerar. Specialtecknen
är inte särskilt många:
\latex|# ^ & _ { } ~ \ % $|

För att använda dem måste man lägga till ett (bakvänt) snedstreck,
\cmd{}, innan tecknet man vill använda (dock ej innan
\cmd{} självt, eftersom \LaTeX{} tolkar
\cmd{\textbackslash} som att man vill tvinga fram en
radbrytning — vill man ha ett \cmd{} använder man istället
\cmd{textbackslash}).
När det gäller \verb|^| och \verb|~| måste man
dessutom lägga till måsvingar efteråt, eftersom dessa tecken normalt används för att
modifiera vokaler. Tecknen måste alltså skrivas på följande sätt:
\latex|\# \^{} \& \_ \{ \} \~{} \textbackslash \% \$|

\section{\LaTeX-kommandon}
För att säga åt \LaTeX{} vad som ska göras används kommandon. Dessa
kommandon följer några enkla regler:
\begin{itemize}
	\item De börjar med ett bakvänd snedstreck (\cmd{}) och följs av sitt
	namn, som bara får innehålla bokstäver%
	\footnote{Det finns även ett antal kommandon som består av \cmd{} och
	exakt en icke-bokstav, till exempel \cmd{\&}.}.
	Kommandon avslutas av ett
	mellanslag, en siffra eller annan godtycklig ”icke-bokstav”.
	
	\item De är skriftlägeskänsliga (\cmd{Kommando} är inte samma sak som
	\cmd{kommando}).
	
	\item Vissa kommandon har även en ”stjärnvariant”, då en stjärna
	(\texttt{*}) läggs till på slutet av kommandonamnet.
\end{itemize}

\LaTeX{} ignorerar tomrum efter ett kommando. Det är därför nödvändigt, om
man vill ha ett mellanslag efter ett kommando, att lägga till antingen en
tom parameter \texttt{\{\}} och ett mellanslag eller ett speciellt
mellanslagskommando (till exempel \texttt{\~{}}) mellan
kommandot och nästa ord. Detta stoppar \LaTeX{} från att äta upp allt
tomrum efter kommandot.

Vissa kommandon kräver en obligatorisk och/eller frivillig parameter som
på ett eller annat sätt bestämmer hur kommandot beter sig. Den generella
strukturen hos ett kommando med sådana parametrar är relativt enkel:
\latex|\kommando[frivillig parameter]{obligatorisk parameter}|

Finns det ingen obligatorisk parameter, eller om den frivilliga parametern
inte används, kan hela biten inklusive hakparantes/måsvinge helt
utelämnas. Ibland kan den frivilliga parametern innehålla många bitar
information, till exempel olika flaggor till paket. Dessa brukar då
separeras med ett kommatecken.


\section{Kommentarer}
Ibland kan det vara bra att kommentera bort (alltså~säga åt \LaTeX{} att
ignorera) vissa bitar av dokumentet. Det kan vara för att förklara
olika kommandon eller definitioner, eller för att ta bort ett stycke man
inte är helt säker på. Detta görs med procenttecknet; när \LaTeX{} stöter
på ett sådant (såvida det inte sitter ett \cmd{} före) så ignorerar den
hela resten av den raden i \TeX{}-filen:
\latex|Lite text\ldots{} % ...och en kommentar|

\section{Grundläggande struktur}
När du nu vet hur \LaTeX{} tolkar tomrum, vad ett kommando är och hur du
kommenterar delar av dina filer är det dags att förklara lite närmre hur
en typisk \LaTeX{}-fil ser ut. Kompilatorn förväntar sig en viss struktur;
till exempel så måste varje dokument inledas med kommandot
\latex|\documentclass{...}|
som berättar för \LaTeX{} vilken dokumentstil du vill använda (det finns
ett antal olika, för till exempel rapporter, böcker eller brev).

Därefter bör man köra kommandon som påverkar hela dokumentet, till
exempel genom att ladda in extra paket eller definiera nya kommandon.
Paket laddar man in med kommandot \cmd{usepackage}:
\latex|\usepackage{...}|

Både \cmd{documentclass} och \cmd{usepackage} tar en obligatorisk parameter (namnet på 
stilen/paketet) och en frivillig parameter som används för att skicka
inställningar till stilen eller paketet som laddas in — en närmare
förklaring av båda dessa kommandon kommer senare.

När allt förarbete är gjort kan man inleda dokumentet. Först berättar man
för \LaTeX{} att innehållet börjar med hjälp av kommandot
\latex|\begin{document}|
och därefter följer dokumentets innehåll. När allt innehåll är slut
avslutar man dokumentet med det snarlika kommandot
\latex|\end{document}|
som säger åt \LaTeX{} att det inte finns något mer som ska typsättas. Vad
man egentligen gör här är att skapa en omgivning med namnet 
\texttt{document}, som betecknar att detta är
dokumentets innehåll. Omgivningar förklaras utförligt på \cpageref{sec:omg}.

\phantomsection
\chapter{Dokumentets layout}\label{sec:1:layout}
\LaTeX{} är inte helt oflexibelt, och man kan med hjälp av olika kommandon
i inledningen till dokumentet ändra både utseendet och beteendet hos
detsamma. Utseendet ändrar man med så kallade dokumentklasser (varje
dokument måste specificera exakt en sådan), medan extra funktionalitet
läggs till i form av så kallade paket.

\section{Dokumentklasser}
Den första informationen \LaTeX{} behöver när den kompilerar ett dokument
är vilken typ av dokument författaren vill skapa. Detta specificeras med
hjälp av kommandot \cmd{documentclass}:
\latex|\documentclass[inställningar]{klass}|

Här berättar \opt{klass} vilken sort dokument som ska skapas, eller vilken
stil som ska användas. Med de flesta \LaTeX{}-distributioner följer ett 
antal standardklasser, och av dessa är det fyra man kan tänkas komma i
kontakt med relativt ofta:
\begin{description}[font=\sffamily]
	\item[article] är en klass för artiklar till tidsskrifter, korta
	rapporter, dokumentation, och annat som inte har någon annan, specifik
	klass. Om du inte vet vilken klass du vill ha, börja med 
	\pack{article}.
	
	\item[report] är en klass för längre rapporter (sådana med flera delar
	eller kapitel), kortare böcker, kandidatrapporter, examensarbeten och
	doktorsavhandlingar.
	
	\item[book] är en klass för riktiga böcker, som ska gå till tryck.
	
	\item[beamer] är en klass för presentationer och overhead-bilder.
\end{description}

Utöver dessa finns det även mer specialiserade klasser, anpassade för
olika publikationer\footnote{Till exempel \pack{elsarticle},
\pack{ieeetran} eller \pack{revtex}}, tillämpningar
\footnote{Exempelvis \pack{refman} eller \pack{sffms}} eller konventioner%
\footnote{Bland annat \pack{eskdx}, som implementerar ryska konventioner}.

De dokumentklasser (eller paket av dokumentklasser) som kan vara
intressanta att titta på om man typsätter dokument efter europeiska
konventioner, eller böcker i allmänhet, är ett fåtal:
\begin{description}
	\item[KOMA-Script] är en samling klasser som är tänkta att vara
	moderna efterträdare till standardklasserna, baserade på europeiska
	standarder. Samlingen innehåller bland annat klasserna \pack{scr­book},
	\pack{scr­reprt} och \pack{scrartcl} som ersätter \pack{book},
	\pack{report} och \pack{article}, respektive.

	\item[\pack{memoir}] är en dokumentklass som skapats för att vara
	väldigt flexibel och möjliggöra drastiska layoutändringar utan att
	behöva skriva mycket \TeX-kod på en låg nivå. Den är främst tänkt
	att ersätta \pack{book} och \pack{report}.

	\item[\pack{octavo}] är tänkt att ersätta \pack{book}, med fokus på
	klassiska design- och layoutprinciper.

	\item[\pack{tufte-book}] är en del av samlingen Tufte-\LaTeX
	och möjliggör typsättning av böcker liknande de som
	statistikern och pionjären inom datavisualisering Edward Tufte
	författade. Klassen är tänkt att användas istället för \pack{book}.
\end{description}
Den intresserade uppmanas att leta upp dokumentationen för dessa på CTAN.

Standardklasserna har också ett antal inställningar som kan ges som en
kommaseparerad lista i den frivilliga parametern till \cmd{documentclass}:
\begin{description}[font=\ttfamily]
	\item[10pt,11pt,12pt] sätter textstorleken (för brödtext) i
	dokumentet. Om ingen av dessa anges är \texttt{10pt} standard.
	
	\item[a4paper,\ldots{}] definierar pappersstorleken. Det finns ett
	antal olika, bland annat \texttt{a5paper} och \texttt{b5paper}, men
	standard är \texttt{letterpaper} — se därför till att ändra,
	eftersom denna pappersstorlek inte används i Sverige.
	
	\item[onecolumn,twocolumn] sätter antalet kolumner i dokumentet.
	Standard är en kolumn (\texttt{onecolumn}),
	och vill man ha två kolumner
	rekommenderas det att man använder paketet \pack{multicol} istället
	för alternativet \texttt{twocolumn}.
	
	\item[twoside,oneside] specificerar om man vill ha ett dubbel- eller 
	enkelsidigt dokument. \pack{article} och \pack{report} är enkelsidiga
	och \pack{book} är dubbelsidig om inget annat anges. Notera att detta
	bara påverkar dokumentets stil; \texttt{twoside} berättar \emph{inte}
	för din skrivare att du vill ha dubbelsidiga utskrifter.
\end{description}

Säg till exempel att du vill skriva en rapport och att du vill använda den
allmänt accepterade textstorleken \SI{12}{\point} på ett A4-papper. Det är
ingen lång rapport, kanske till och med en inlämningsuppgift, och du
bestämmer dig därför för att använda \pack{article}-klassen:
\latex|\documentclass[12pt,a4paper]{article}|

\section{Paket}
När du skriver ett dokument kommer du troligtvis att märka att vissa saker
är svåra eller ointuitiva (kanske rent av omöjliga\footnote{Tekniskt sett
är de inte omöjliga, eftersom \TeX{} är Turingkomplett. Nästan alla paket
är implementerade med \LaTeX-kod.}) att göra med grundläggande \LaTeX-kod.
Lyckligvis finns det då nästan alltid ett paket på CTAN som förenklar
det du vill göra. Installerade paket kan inkluderas med kommandot
\cmd{usepackage}:
\latex|\usepackage[inställningar]{paketnamn}|

Notera dock att du måste installera dessa paket innan du kan inkludera dem
i ditt \LaTeX-dokument. Din \LaTeX-distribution kommer förmodligen med de
flesta paket förinstallerade, men om du får ett felmeddelande när du
försöker använda ett paket beror det troligtvis på att det inte är
installerat. I \TeX{} Live och Mac\TeX{} kan man installera nya paket med
kommandot \cli{tlmgr install \opt{paketnamn}}.
De flesta paketen kommer även med dokumentation, som kan nås med kommandot
\cli{texdoc}, men ibland kan det vara lättare att leta upp motsvarande
paket på CTAN och kolla på dokumentationen där istället, eller använda
online-tjänsten \TeX{}doc.net\footnote{Man kan då helt enkelt gå till
\href{http://texdoc.net}{\nolinkurl{http://texdoc.net/pkg/<paketnamn>}}
för att visa dokumentationen för ett paket.}.

Några paket finns alltid och är mycket viktiga eftersom de berättar för
\LaTeX{} hur in- och utdata (ska) formateras:
\begin{description}[font=\sffamily]
	\label{pack:fontenc}
	\item[fontenc] berättar för \LaTeX{} vilken sorts typsnitt det
	typsatta dokumentet ska använda; de vanligaste är \texttt{T3}, \texttt{OT1} och
	\texttt{T1} — oftast vill man ha \texttt{T1}, som är vektorbaserad:
	\latex|\usepackage[T1]{fontenc}|
	
	\label{pack:inputenx}
	\item[inputenx] berättar för \LaTeX{} vilken teckenkodning indatan har
	skrivits i. \LaTeX{} (som
	stammar från 80-talet) antar att teckenkodningen är
	\textsc{iso-8859-1} om inget annat anges, men många moderna system
	använder \UTF vilket man då måste berätta:
	\latex|\usepackage[utf8]{inputenx}|
	\label{sec:1:inputenx}
\end{description}

\phantomsection
\chapter{Från \TeX{} till \PDF}
För att skapa ett typsatt \PDF-dokument utifrån en \LaTeX-fil måste man
köra den genom den så kallade kompilatorn. \LaTeX{} kommer inte med något
fint GUI med behändiga knappar att trycka på\footnote{Även om det finns
sådana verktyg, till exempel \LyX{} och \TeX{}nicCenter.}; man måste
istället köra kompilatorn från terminalen.

Normalt måste man köra kompilatorn åtminstone två gånger, så att \LaTeX{}
får en chans att förbättra och korrigera småsaker som referenser,
innehållsförteckningar med mera\footnote{Detta är en konsekvens av hur
\TeX{} fungerar; kompilatorn läser filen bit för bit och expanderar kommandon —
det går inte att röra sig ”bakåt” i filen, så vill man ändra på något
man redan gått förbi måste man spara information om detta i en extern fil
som man läser in under nästa körning.}.
Kompilatorn brukar varna om detta och
be om ännu en körning. Använder man vissa externa verktyg som \BibTeX{}
måste man även köra detta program; arbetsflödet blir då 
\LaTeX\hspace{0pt}{\utffont →}\hspace{0pt}\BibTeX\hspace{0pt}{\utffont →}\hspace{0pt}\LaTeX\hspace{0pt}{\utffont →}\hspace{0pt}\LaTeX.

\section{Kompilatorn: \pdfLaTeX}
Det finns ett antal olika kompilatorer att tillgå (den vanligaste heter
bara \cli{latex} och genererar \DVI-filer), men den som är att föredra är
\pdfLaTeX{}. Fördelen med denna kompilator är att den har utökat stöd för
vissa mikrotypografiska funktioner, till exempel hängande punktuation
\parencite{Thanh00}, men framför allt så kompilerar den dina \LaTeX-filer till
\PDF-dokument, istället för de \DVI-filer \cli{latex} producerar.

Användningen är enkel; när du vill kompilera ditt dokument (vilket du som
sagt vill göra några gånger i rad) anropar du helt enkelt programmet
\cli{pdflatex} (testa gärna på koden i \cref{ex:initio}):
\mint{sh}|pdflatex filnamn.tex|

Detta gör att \pdfLaTeX{} kompilerar ditt program och skapar ett antal
filer, däribland \texttt{filnamn.pdf}, som är det slutgiltiga (eller, så
slutgiltigt \LaTeX{} hunnit göra det) \PDF-dokumentet.

\begin{kod}[tbp]
	\begin{latexcode}
\documentclass[a4paper,12pt]{article}
\usepackage[utf8]{inputenx}
\usepackage[T1]{fontenc}

\begin{document}
Det här är ett första stycke, ett exempel
på hur    text    typsätts    av \LaTeX{}.

Oj, ett nytt stycke! (Två stycken i samma
dokument, kan du tänka dig?)
\end{document}
	\end{latexcode}
	\caption{Ett mycket enkelt \LaTeX-dokument med endast det allra
	nödvändigaste.}
	\label{ex:initio}
\end{kod}

\section{Automatisera med \cli{latexmk}}
Det kan vara tröttsamt att konstant kompilera sina filer och hålla koll på
hur många gånger man ska köra \LaTeX{} och andra verktyg. Lyckligtvis 
finns det sätt att automatisera processen, och ett av det enklaste är
Perl-skriptet \cli{latexmk}, som finns tillgängligt på CTAN%
\footnote{\url{http://www.ctan.org/tex-archive/support/latexmk/}}.
Skriptet kör \LaTeX{} och tolkar loggen som skapas för att avgöra om det
behövs fler körningar, och kan även konfigureras för att kontinuerligt
kolla efter ändringar i \LaTeX-filen (och andra berörda filer) och
kompilera om hela dokumentet när dessa ändras.

Att köra \cli{latexmk} är lika lätt som att köra \cli{pdflatex}:
\mint{sh}|latexmk -pdf filnamn.tex|

\chapter{Filer du kanske stöter på}
När man arbetar med \LaTeX{} finner man ganska snabbt att det dyker upp en
stor mängd olika filer (därför bör man också ha varje dokument i en egen
mapp) med mer eller mindre uppenbara filändelser. Vissa behövs för att
\LaTeX{} ska fungera, vissa genereras av \LaTeX{} och används i senare
körningar och vissa är relativt överflödiga.

Eftersom olika paket också kan skapa filer av godtycklig typ kommer listan
nedan inte vara fullständig, men förhoppningsvis innehåller den de filer
man oftast stöter på i sitt arbete med \LaTeX{}.

\subsection*{Filer som används av \LaTeX{}}
\begin{description}[font=\ttfamily]
	\item[.tex] Ett \TeX- eller \LaTeX-dokument. Kompileras med
	\cli{pdflatex} (eller \cli{latex}, om man vill ha \DVI-filer).
	
	\item[.sty] Ett \LaTeX-paket, som kan inkluderas med \cli{usepackage}.
	
	\item[.dtx] Dokumenterad \TeX-kod. Hittar du en sån här fil så är det
	troligtvis ett distribuerat paket, och det borde följa med en fil med
	ändelsen \texttt{.ins}. Om du kör \LaTeX{} på en \texttt{.dtx}-fil så
	kommer dokumentationen för motsvarande \LaTeX-paket genereras.
	
	\item[.ins] Installationsfil för en motsvarande \texttt{.dtx}-fil. Kör
	man \LaTeX{} på denna så kommer den oftast att generera en 
	\texttt{.sty}- eller \texttt{.cls}-fil (detta beror så klart på vad
	paketet i fråga innehåller).
	
	\item[.cls] En dokumentklass. Denna kan väljas med 
	\cli{documentclass}.
\end{description}

\subsection*{Filer som genereras av \LaTeX{}}
\begin{description}[font=\ttfamily]
	\item[.pdf] \PDF-dokument \eng{Portable Document File}. Det här är
	sannolikt den fil \LaTeX{} genererat från din \texttt{.tex}-fil,
	förutsatt att du inte lagrar andra \PDF-dokument i samma mapp (vilket
	du inte bör göra).
	
	\item[.log] En detaljerad logg som berättar vad som hände under den
	senaste körningen av \cli{pdflatex}.
	
	\item[.toc] Lagrar alla rubriker, och läses in av \LaTeX{} under nästa
	körning då den genererar en innehållsförteckning (om en sådan ska 
	finnas med i dokumentet).
	
	\item[.lof] Som \texttt{.toc}, fast med en lista över figurer.
	
	\item[.lot] Som \texttt{.toc} och \texttt{.lof}, fast med en lista 
	över tabeller.
	
	\item[.aux] En fil som innehåller information om korsreferenser,
	etiketter och annat. Läses in vid nästa körning av \cli{pdflatex}.
\end{description}
\end{document}

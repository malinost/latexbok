\ifdefined\latexbokFontdir\else\def\latexbokFontdir{../../fonts}\fi
\ifdefined\latexbokFiguredir\else\def\latexbokFiguredir{../../examples}\fi
\documentclass[lang=sv,ptsize=10pt,font=none,nomath,titles=bf,../../a4.tex]{subfiles}
\begin{document}
%% NOTE: run in nonstopmode to ignore XeTeX bug
%% See http://tex.stackexchange.com/questions/122923/xelatex-problem-missing-chars-in-texlive-2013-textfont-xxx-is-undefined-erro
%% Also, http://tex.stackexchange.com/questions/91592/where-to-find-official-and-extended-documentation-for-tex-latexs-commandlin
\chardef\previousmode=\interactionmode
\nonstopmode
%% END NOTE
%
\section{Referenser med \pack{biblatex}}\label{sec:5}
\index{bibliografi}
En stor fördel med \LaTeX{} är att man kan automatisera hanteringen av
till exempel referenser. Paketet \pack{biblatex}, tillsammans med
programmet \cli{biber}\footnote{Tekniskt sett kan även
\BibTeX{} användas tillsammans med \pack{biblatex}, men det finns egentligen ingen anledning att göra det. Den som är intresserad av
\BibTeX{}, det gamla systemet för att hantera referenser, hänvisas till introduktionen i \textcite{Fenn06} samt \BibTeX{}-manualen
\parencite{Patashnik88a}.}, gör detta mycket enkelt genom att låta dig
samla alla dina referenser i en extern databas, som man sedan läser in
och hänvisar till. Denna databasen lagras i ett textbaserat format, så
den går att hantera med samma program du skriver \LaTeX-kod i, eller i
ett specialiserat program\footnote{Exempel på sådana är \emph{JabRef},
\emph{BibDesk} och \emph{refbase}.}.

Användningen av \pack{biblatex}-\cli{biber} (där \pack{biblatex} är det
\LaTeX-paket som hanterar typsättningen, och \cli{biber} är ett externt
program som behandlar och sorterar referenserna) kan delas upp i några
olika delar. Till att börja med kommer vi att diskutera formatet av
själva referensdatabasen. Därefter diskuteras paketet \pack{biblatex},
några av dess viktigaste inställningar och stilar, och till sist
sammanfattas användningen av de två tillsammans, speciellt hur man
refererar och genererar referenslistan.

\subsection{Referensdatabaser}
\index{syntax!BibTeX@\hologo{BibTeX}|(}
Databaserna \pack{biblatex} använder är, som nämndes tidigare, helt
vanliga textfiler. Dessa innehåller ett antal block, ett per referens
i databasen, som i sin tur innehåller fält med information om varje
referens. Ett typiskt block innehåller en nyckel (som används när man
sedan i \LaTeX-dokumentet refererar till referensen), en titel, en
författare och ett årtal. Beroende på vilken typ av referens (bok,
artikel, och så vidare) det handlar om kan även andra fält finnas med.

\begin{kod}[tbp]
	\centering
	\vfil\cprotect[mm]\ifdraft{\inputminted[frame=single]{latex}{\latexbokFiguredir/5/bibdata.bib}}{\inputminted[frame=single,bgcolor=mintedbg,rulecolor=\color{mintedbg}]{latex}{\latexbokFiguredir/5/bibdata.bib}}\vfil
	\caption{En enkel exempelreferens ur en referensdatabas.}
	\label{ex:bibtex}
\end{kod}

\index{referens!-databas}
\Cref{ex:bibtex} visar ett block ur en referensdatabas. Blocket inleds
med en blocktyp (\texttt{@article}), som indikerar vilken typ referensen
har, och den nyckel som används när man refererar till källan
(\texttt{Friedman01}). Därefter följer ett antal självförklarande
fält som lagrar information om referensen, som i det här fallet är en
artikel ur en tidsskrift. En del av dessa fält är obligatoriska (bland
annat titel och författare), medan andra är frivilliga (till exempel
\texttt{month} och \texttt{doi}). \Cref{tab:blocktyper} listar de några
blocktyper och en del av de obligatoriska/frivilliga fält som hör till.
En fullständig förteckning finns i \textcite[7\psqq]{Lehman13}.
\index{syntax!BibTeX@\hologo{BibTeX}|)}

\begin{table}[p]
	\begin{minipage}[][0.98\textheight][t]{0.98\textwidth}
		\newlength{\oldfboxsep}\setlength{\oldfboxsep}{\fboxsep}
		\setlength{\fboxsep}{0pt}
		\def\r{% (fold)
			\setlength{\fboxsep}{\oldfboxsep}%
			\colorbox{required}{\color{required}o}%\checkmark}%
		} % (end)
		\def\o{% (fold)
			\setlength{\fboxsep}{\oldfboxsep}%
			\colorbox{optional}{\color{optional}v}%\Large$\circ$}%
		} % (end)
		\def\u{% (fold)
			\setlength{\fboxsep}{\oldfboxsep}%
			\colorbox{unavailable}{\color{unavailable}x}%$\times$}%
		} % (end)
		\centering
		\caption{\setlength{\oldfboxsep}{0.5\oldfboxsep}%
			Några av referenstyperna i \pack{biblatex} och en del av
			deras tillhörande fält.
			Både typnamn och fältnamn är relativt självförklarande,
			men se \textcite[7\psqq]{Lehman13} för mer information.
			\emph{Nyckel:} \fbox{\r} obligatoriskt,
			\fbox{\o} valfritt och
			\fbox{\u} otillgängligt fält.
		}
		\label{tab:blocktyper}
		\NewCoffin\TableCoffin
		\SetHorizontalCoffin\TableCoffin{
			\ExplSyntaxOn
\msg_new:nnn{latexbok}{required-hidden}{A~required~field~was~hidden!}
\newcolumntype{/}{c@{\,}}
\newcolumntype{H}{>{\setbox0=\hbox\bgroup\def\r{\msg_error:nn{latexbok}{required-hidden}}}c<{\egroup}@{}}
\ExplSyntaxOff
\begin{tabular}{lHHHH/HHH/H/HH////HHHHHHHHHHHHHHHHHHHH//% hyphenation
				HHH/HH/HHH/HHH/HH/H/HHH/H/HH/HHHHH/HHHH/% publisher
				HHHHHH/HHHHHHHHHHHH//HH///H///HH/}% year
	\toprule
	Blocktyp & \rotatebox{90}{\texttt{abstract}} & \rotatebox{90}{\texttt{addendum}} & \rotatebox{90}{\texttt{afterword}} & \rotatebox{90}{\texttt{annotation}} & \rotatebox{90}{\texttt{author}} & \rotatebox{90}{\texttt{authortype}} & \rotatebox{90}{\texttt{bookauthor}} & \rotatebox{90}{\texttt{booksubtitle}} & \rotatebox{90}{\texttt{booktitle}} & \rotatebox{90}{\texttt{booktitleaddon}} & \rotatebox{90}{\texttt{chapter}} & \rotatebox{90}{\texttt{commentator}} & \rotatebox{90}{\texttt{crossref}} & \rotatebox{90}{\texttt{date}\footnote{\label{fn:year-date}Antingen \texttt{month}/\texttt{year} (där \texttt{month} är valfri) \emph{eller} \texttt{date} ska specificeras, ej båda två.}} & \rotatebox{90}{\texttt{doi}} & \rotatebox{90}{\texttt{edition}} & \rotatebox{90}{\texttt{editor}} & \rotatebox{90}{\texttt{editora}} & \rotatebox{90}{\texttt{editoratype}} & \rotatebox{90}{\texttt{editorb}} & \rotatebox{90}{\texttt{editorbtype}} & \rotatebox{90}{\texttt{editorc}} & \rotatebox{90}{\texttt{editorctype}} & \rotatebox{90}{\texttt{editortype}} & \rotatebox{90}{\texttt{eid}} & \rotatebox{90}{\texttt{entryset}} & \rotatebox{90}{\texttt{entrysubtype}} & \rotatebox{90}{\texttt{eprint}} & \rotatebox{90}{\texttt{eprintclass}} & \rotatebox{90}{\texttt{eprinttype}} & \rotatebox{90}{\texttt{eventdate}} & \rotatebox{90}{\texttt{eventtitle}} & \rotatebox{90}{\texttt{execute}} & \rotatebox{90}{\texttt{file}} & \rotatebox{90}{\texttt{foreword}} & \rotatebox{90}{\texttt{gender}} & \rotatebox{90}{\texttt{holder}} & \rotatebox{90}{\texttt{howpublished}} & \rotatebox{90}{\texttt{hyphenation}} & \rotatebox{90}{\texttt{ids}} & \rotatebox{90}{\texttt{indexsorttitle}} & \rotatebox{90}{\texttt{indextitle}} & \rotatebox{90}{\texttt{institution}} & \rotatebox{90}{\texttt{introduction}} & \rotatebox{90}{\texttt{isan}} & \rotatebox{90}{\texttt{isbn}} & \rotatebox{90}{\texttt{ismn}} & \rotatebox{90}{\texttt{isrn}} & \rotatebox{90}{\texttt{issn}} & \rotatebox{90}{\texttt{issue}} & \rotatebox{90}{\texttt{issuetitle}} & \rotatebox{90}{\texttt{iswc}} & \rotatebox{90}{\texttt{journalsubtitle}} & \rotatebox{90}{\texttt{journaltitle}} & \rotatebox{90}{\texttt{keywords}} & \rotatebox{90}{\texttt{label}} & \rotatebox{90}{\texttt{language}} & \rotatebox{90}{\texttt{library}} & \rotatebox{90}{\texttt{location}} & \rotatebox{90}{\texttt{mainsubtitle}} & \rotatebox{90}{\texttt{maintitle}} & \rotatebox{90}{\texttt{maintitleaddon}} & \rotatebox{90}{\texttt{month}\hyperref[fn:year-date]{${}^\text{\ref{fn:year-date}}$}} & \rotatebox{90}{\texttt{nameaddon}} & \rotatebox{90}{\texttt{note}} & \rotatebox{90}{\texttt{number}} & \rotatebox{90}{\texttt{options}} & \rotatebox{90}{\texttt{organization}} & \rotatebox{90}{\texttt{origdate}} & \rotatebox{90}{\texttt{origlanguage}} & \rotatebox{90}{\texttt{origlocation}} & \rotatebox{90}{\texttt{origpublisher}} & \rotatebox{90}{\texttt{origtitle}} & \rotatebox{90}{\texttt{pages}} & \rotatebox{90}{\texttt{pagetotal}} & \rotatebox{90}{\texttt{pagination}} & \rotatebox{90}{\texttt{part}} & \rotatebox{90}{\texttt{presort}} & \rotatebox{90}{\texttt{publisher}} & \rotatebox{90}{\texttt{pubstate}} & \rotatebox{90}{\texttt{related}} & \rotatebox{90}{\texttt{relatedoptions}} & \rotatebox{90}{\texttt{relatedstring}} & \rotatebox{90}{\texttt{relatedtype}} & \rotatebox{90}{\texttt{reprinttitle}} & \rotatebox{90}{\texttt{series}} & \rotatebox{90}{\texttt{shortauthor}} & \rotatebox{90}{\texttt{shorteditor}} & \rotatebox{90}{\texttt{shorthand}} & \rotatebox{90}{\texttt{shorthandintro}} & \rotatebox{90}{\texttt{shortjournal}} & \rotatebox{90}{\texttt{shortseries}} & \rotatebox{90}{\texttt{shorttitle}} & \rotatebox{90}{\texttt{sortkey}} & \rotatebox{90}{\texttt{sortname}} & \rotatebox{90}{\texttt{sortshorthand}} & \rotatebox{90}{\texttt{sorttitle}} & \rotatebox{90}{\texttt{sortyear}} & \rotatebox{90}{\texttt{subtitle}} & \rotatebox{90}{\texttt{title}} & \rotatebox{90}{\texttt{titleaddon}} & \rotatebox{90}{\texttt{translator}} & \rotatebox{90}{\texttt{type}} & \rotatebox{90}{\texttt{url}} & \rotatebox{90}{\texttt{urldate}} & \rotatebox{90}{\texttt{venue}} & \rotatebox{90}{\texttt{version}} & \rotatebox{90}{\texttt{volume}} & \rotatebox{90}{\texttt{volumes}} & \rotatebox{90}{\texttt{xdata}} & \rotatebox{90}{\texttt{xref}} & \rotatebox{90}{\texttt{year}\hyperref[fn:year-date]{${}^\text{\ref{fn:year-date}}$}} \\
	\midrule
	\texttt{@article} & \u & \o & \u & \u & \r & \u & \u & \u & \u & \u & \u & \o & \o & \r & \o & \u & \o & \o & \u & \o & \u & \o & \u & \u & \o & \o & \u & \o & \o & \o & \u & \u & \o & \u & \u & \o & \u & \u & \o & \o & \o & \o & \u & \u & \u & \u & \u & \u & \o & \o & \o & \u & \o & \r & \o & \o & \o & \o & \u & \u & \u & \u & \o & \u & \o & \o & \o & \u & \u & \o & \u & \u & \u & \o & \u & \u & \u & \o & \u & \o & \o & \o & \o & \o & \o & \o & \o & \o & \o & \o & \u & \u & \u & \o & \o & \o & \o & \o & \o & \r & \o & \o & \u & \o & \o & \u & \o & \o & \u & \o & \o & \r \\
	\texttt{@book} & \u & \o & \o & \u & \r & \u & \u & \u & \u & \u & \o & \o & \o & \r & \o & \o & \o & \o & \u & \o & \u & \o & \u & \u & \u & \o & \u & \o & \o & \o & \u & \u & \o & \u & \o & \o & \u & \u & \o & \o & \o & \o & \u & \o & \u & \o & \u & \u & \u & \u & \u & \u & \u & \u & \o & \o & \o & \o & \o & \o & \o & \o & \u & \u & \o & \o & \o & \u & \u & \o & \u & \u & \u & \o & \o & \u & \o & \o & \o & \o & \o & \o & \o & \o & \o & \o & \o & \o & \o & \o & \u & \u & \u & \o & \o & \o & \o & \o & \o & \r & \o & \o & \u & \o & \o & \u & \u & \o & \o & \o & \o & \r \\
	\texttt{@mvbook} & \u & \o & \o & \u & \r & \u & \u & \u & \u & \u & \u & \o & \o & \r & \o & \o & \o & \o & \u & \o & \u & \o & \u & \u & \u & \o & \u & \o & \o & \o & \u & \u & \o & \u & \o & \o & \u & \u & \o & \o & \o & \o & \u & \o & \u & \o & \u & \u & \u & \u & \u & \u & \u & \u & \o & \o & \o & \o & \o & \u & \u & \u & \u & \u & \o & \o & \o & \u & \u & \o & \u & \u & \u & \u & \o & \u & \u & \o & \o & \o & \o & \o & \o & \o & \o & \o & \o & \o & \o & \o & \u & \u & \u & \o & \o & \o & \o & \o & \o & \r & \o & \o & \u & \o & \o & \u & \u & \u & \o & \o & \o & \r \\
	\texttt{@inbook} & \u & \o & \o & \u & \r & \u & \o & \o & \r & \o & \o & \o & \o & \r & \o & \o & \o & \o & \u & \o & \u & \o & \u & \u & \u & \o & \u & \o & \o & \o & \u & \u & \o & \u & \o & \o & \u & \u & \o & \o & \o & \o & \u & \o & \u & \o & \u & \u & \u & \u & \u & \u & \u & \u & \o & \o & \o & \o & \o & \o & \o & \o & \u & \u & \o & \o & \o & \u & \u & \o & \u & \u & \u & \o & \u & \u & \o & \o & \o & \o & \o & \o & \o & \o & \o & \o & \o & \o & \o & \o & \u & \u & \u & \o & \o & \o & \o & \o & \o & \r & \o & \o & \u & \o & \o & \u & \u & \o & \o & \o & \o & \r \\
	\texttt{@bookinbook} & \u & \o & \o & \u & \r & \u & \o & \o & \r & \o & \o & \o & \o & \r & \o & \o & \o & \o & \u & \o & \u & \o & \u & \u & \u & \o & \u & \o & \o & \o & \u & \u & \o & \u & \o & \o & \u & \u & \o & \o & \o & \o & \u & \o & \u & \o & \u & \u & \u & \u & \u & \u & \u & \u & \o & \o & \o & \o & \o & \o & \o & \o & \u & \u & \o & \o & \o & \u & \u & \o & \u & \u & \u & \o & \u & \u & \o & \o & \o & \o & \o & \o & \o & \o & \o & \o & \o & \o & \o & \o & \u & \u & \u & \o & \o & \o & \o & \o & \o & \r & \o & \o & \u & \o & \o & \u & \u & \o & \o & \o & \o & \r \\
	\texttt{@suppbook} & \u & \o & \o & \u & \r & \u & \o & \o & \r & \o & \o & \o & \o & \r & \o & \o & \o & \o & \u & \o & \u & \o & \u & \u & \u & \o & \u & \o & \o & \o & \u & \u & \o & \u & \o & \o & \u & \u & \o & \o & \o & \o & \u & \o & \u & \o & \u & \u & \u & \u & \u & \u & \u & \u & \o & \o & \o & \o & \o & \o & \o & \o & \u & \u & \o & \o & \o & \u & \u & \o & \u & \u & \u & \o & \u & \u & \o & \o & \o & \o & \o & \o & \o & \o & \o & \o & \o & \o & \o & \o & \u & \u & \u & \o & \o & \o & \o & \o & \o & \r & \o & \o & \u & \o & \o & \u & \u & \o & \o & \o & \o & \r \\
	\texttt{@booklet}\footnote{\label{fn:author-editor}Antingen \texttt{author} \emph{eller} \texttt{editor} ska specificeras, ej båda två.} & \u & \o & \u & \u & \r & \u & \u & \u & \u & \u & \o & \u & \o & \r & \o & \u & \r & \u & \u & \u & \u & \u & \u & \u & \u & \o & \u & \o & \o & \o & \u & \u & \o & \u & \u & \o & \u & \o & \o & \o & \o & \o & \u & \u & \u & \u & \u & \u & \u & \u & \u & \u & \u & \u & \o & \o & \o & \o & \o & \u & \u & \u & \u & \u & \o & \u & \o & \u & \u & \u & \u & \u & \u & \o & \o & \u & \u & \o & \u & \o & \o & \o & \o & \o & \o & \u & \o & \o & \o & \o & \u & \u & \u & \o & \o & \o & \o & \o & \o & \r & \o & \u & \o & \o & \o & \u & \u & \u & \u & \o & \o & \r \\
	\texttt{@collection} & \u & \o & \o & \u & \u & \u & \u & \u & \u & \u & \o & \o & \o & \r & \o & \o & \r & \o & \u & \o & \u & \o & \u & \u & \u & \o & \u & \o & \o & \o & \u & \u & \o & \u & \o & \o & \u & \u & \o & \o & \o & \o & \u & \o & \u & \o & \u & \u & \u & \u & \u & \u & \u & \u & \o & \o & \o & \o & \o & \o & \o & \o & \u & \u & \o & \o & \o & \u & \u & \o & \u & \u & \u & \o & \o & \u & \o & \o & \o & \o & \o & \o & \o & \o & \o & \o & \o & \o & \o & \o & \u & \u & \u & \o & \o & \o & \o & \o & \o & \r & \o & \o & \u & \o & \o & \u & \u & \o & \o & \o & \o & \r \\
	\texttt{@mvcollection} & \u & \o & \o & \u & \u & \u & \u & \u & \u & \u & \u & \o & \o & \r & \o & \o & \r & \o & \u & \o & \u & \o & \u & \u & \u & \o & \u & \o & \o & \o & \u & \u & \o & \u & \o & \o & \u & \u & \o & \o & \o & \o & \u & \o & \u & \o & \u & \u & \u & \u & \u & \u & \u & \u & \o & \o & \o & \o & \o & \u & \u & \u & \u & \u & \o & \o & \o & \u & \u & \o & \u & \u & \u & \u & \o & \u & \u & \o & \o & \o & \o & \o & \o & \o & \o & \o & \o & \o & \o & \o & \u & \u & \u & \o & \o & \o & \o & \o & \o & \r & \o & \o & \u & \o & \o & \u & \u & \u & \o & \o & \o & \r \\
	\texttt{@incollection} & \u & \o & \o & \u & \r & \u & \u & \o & \r & \o & \o & \o & \o & \r & \o & \o & \r & \o & \u & \o & \u & \o & \u & \u & \u & \o & \u & \o & \o & \o & \u & \u & \o & \u & \o & \o & \u & \u & \o & \o & \o & \o & \u & \o & \u & \o & \u & \u & \u & \u & \u & \u & \u & \u & \o & \o & \o & \o & \o & \o & \o & \o & \u & \u & \o & \o & \o & \u & \u & \o & \u & \u & \u & \o & \u & \u & \o & \o & \o & \o & \o & \o & \o & \o & \o & \o & \o & \o & \o & \o & \u & \u & \u & \o & \o & \o & \o & \o & \o & \r & \o & \o & \u & \o & \o & \u & \u & \o & \o & \o & \o & \r \\
	\texttt{@suppcollection} & \u & \o & \o & \u & \r & \u & \u & \o & \r & \o & \o & \o & \o & \r & \o & \o & \r & \o & \u & \o & \u & \o & \u & \u & \u & \o & \u & \o & \o & \o & \u & \u & \o & \u & \o & \o & \u & \u & \o & \o & \o & \o & \u & \o & \u & \o & \u & \u & \u & \u & \u & \u & \u & \u & \o & \o & \o & \o & \o & \o & \o & \o & \u & \u & \o & \o & \o & \u & \u & \o & \u & \u & \u & \o & \u & \u & \o & \o & \o & \o & \o & \o & \o & \o & \o & \o & \o & \o & \o & \o & \u & \u & \u & \o & \o & \o & \o & \o & \o & \r & \o & \o & \u & \o & \o & \u & \u & \o & \o & \o & \o & \r \\
	\texttt{@manual}\hyperref[fn:author-editor]{${}^\text{\ref{fn:author-editor}}$} & \u & \o & \u & \u & \r & \u & \u & \u & \u & \u & \o & \u & \o & \r & \o & \o & \r & \u & \u & \u & \u & \u & \u & \u & \u & \o & \u & \o & \o & \o & \u & \u & \o & \u & \u & \o & \u & \u & \o & \o & \o & \o & \u & \u & \u & \o & \u & \u & \u & \u & \u & \u & \u & \u & \o & \o & \o & \o & \o & \u & \u & \u & \u & \u & \o & \o & \o & \o & \u & \u & \u & \u & \u & \o & \o & \u & \u & \o & \o & \o & \o & \o & \o & \o & \o & \o & \o & \o & \o & \o & \u & \u & \u & \o & \o & \o & \o & \o & \o & \r & \o & \u & \o & \o & \o & \u & \o & \u & \u & \o & \o & \r \\
	\texttt{@misc}\hyperref[fn:author-editor]{${}^\text{\ref{fn:author-editor}}$} & \u & \o & \u & \u & \r & \u & \u & \u & \u & \u & \u & \u & \o & \r & \o & \u & \r & \u & \u & \u & \u & \u & \u & \u & \u & \o & \u & \o & \o & \o & \u & \u & \o & \u & \u & \o & \u & \o & \o & \o & \o & \o & \u & \u & \u & \u & \u & \u & \u & \u & \u & \u & \u & \u & \o & \o & \o & \o & \o & \u & \u & \u & \o & \u & \o & \u & \o & \o & \u & \u & \u & \u & \u & \u & \u & \u & \u & \o & \u & \o & \o & \o & \o & \o & \o & \u & \o & \o & \o & \o & \u & \u & \u & \o & \o & \o & \o & \o & \o & \r & \o & \u & \o & \o & \o & \u & \o & \u & \u & \o & \o & \r \\
	\texttt{@online}\hyperref[fn:author-editor]{${}^\text{\ref{fn:author-editor}}$} & \u & \o & \u & \u & \r & \u & \u & \u & \u & \u & \u & \u & \o & \r & \u & \u & \r & \u & \u & \u & \u & \u & \u & \u & \u & \o & \u & \u & \u & \u & \u & \u & \o & \u & \u & \o & \u & \u & \o & \o & \o & \o & \u & \u & \u & \u & \u & \u & \u & \u & \u & \u & \u & \u & \o & \o & \o & \o & \u & \u & \u & \u & \o & \u & \o & \u & \o & \o & \u & \u & \u & \u & \u & \u & \u & \u & \u & \o & \u & \o & \o & \o & \o & \o & \o & \u & \o & \o & \o & \o & \u & \u & \u & \o & \o & \o & \o & \o & \o & \r & \o & \u & \u & \r & \o & \u & \o & \u & \u & \o & \o & \r \\
	\texttt{@patent} & \u & \o & \u & \u & \r & \u & \u & \u & \u & \u & \u & \u & \o & \r & \o & \u & \u & \u & \u & \u & \u & \u & \u & \u & \u & \o & \u & \o & \o & \o & \u & \u & \o & \u & \u & \o & \o & \u & \o & \o & \o & \o & \u & \u & \u & \u & \u & \u & \u & \u & \u & \u & \u & \u & \o & \o & \u & \o & \o & \u & \u & \u & \o & \u & \o & \r & \o & \u & \u & \u & \u & \u & \u & \u & \u & \u & \u & \o & \u & \o & \o & \o & \o & \o & \o & \u & \o & \o & \o & \o & \u & \u & \u & \o & \o & \o & \o & \o & \o & \r & \o & \u & \o & \o & \o & \u & \o & \u & \u & \o & \o & \r \\
	\texttt{@periodical} & \u & \o & \u & \u & \u & \u & \u & \u & \u & \u & \u & \u & \o & \r & \o & \u & \r & \o & \u & \o & \u & \o & \u & \u & \u & \o & \u & \o & \o & \o & \u & \u & \o & \u & \u & \o & \u & \u & \o & \o & \o & \o & \u & \u & \u & \u & \u & \u & \o & \o & \o & \u & \u & \u & \o & \o & \o & \o & \u & \u & \u & \u & \o & \u & \o & \o & \o & \u & \u & \u & \u & \u & \u & \u & \u & \u & \u & \o & \u & \o & \o & \o & \o & \o & \o & \o & \o & \o & \o & \o & \u & \u & \u & \o & \o & \o & \o & \o & \o & \r & \u & \u & \u & \o & \o & \u & \u & \o & \u & \o & \o & \r \\
	\texttt{@suppperiodical} & \u & \o & \u & \u & \r & \u & \u & \u & \u & \u & \u & \o & \o & \r & \o & \u & \o & \o & \u & \o & \u & \o & \u & \u & \o & \o & \u & \o & \o & \o & \u & \u & \o & \u & \u & \o & \u & \u & \o & \o & \o & \o & \u & \u & \u & \u & \u & \u & \o & \o & \o & \u & \o & \r & \o & \o & \o & \o & \u & \u & \u & \u & \o & \u & \o & \o & \o & \u & \u & \o & \u & \u & \u & \o & \u & \u & \u & \o & \u & \o & \o & \o & \o & \o & \o & \o & \o & \o & \o & \o & \u & \u & \u & \o & \o & \o & \o & \o & \o & \r & \o & \o & \u & \o & \o & \u & \o & \o & \u & \o & \o & \r \\
	\texttt{@proceedings} & \u & \o & \u & \u & \u & \u & \u & \u & \u & \u & \o & \u & \o & \r & \o & \u & \r & \u & \u & \u & \u & \u & \u & \u & \u & \o & \u & \o & \o & \o & \o & \o & \o & \u & \u & \o & \u & \u & \o & \o & \o & \o & \u & \u & \u & \o & \u & \u & \u & \u & \u & \u & \u & \u & \o & \o & \o & \o & \o & \o & \o & \o & \o & \u & \o & \o & \o & \o & \u & \u & \u & \u & \u & \o & \o & \u & \o & \o & \o & \o & \o & \o & \o & \o & \o & \o & \o & \o & \o & \o & \u & \u & \u & \o & \o & \o & \o & \o & \o & \r & \o & \u & \u & \o & \o & \o & \u & \o & \o & \o & \o & \r \\
	\texttt{@mvproceedings} & \u & \o & \u & \u & \u & \u & \u & \u & \u & \u & \u & \u & \o & \r & \o & \u & \r & \u & \u & \u & \u & \u & \u & \u & \u & \o & \u & \o & \o & \o & \o & \o & \o & \u & \u & \o & \u & \u & \o & \o & \o & \o & \u & \u & \u & \o & \u & \u & \u & \u & \u & \u & \u & \u & \o & \o & \o & \o & \o & \u & \u & \u & \o & \u & \o & \o & \o & \o & \u & \u & \u & \u & \u & \u & \o & \u & \u & \o & \o & \o & \o & \o & \o & \o & \o & \o & \o & \o & \o & \o & \u & \u & \u & \o & \o & \o & \o & \o & \o & \r & \o & \u & \u & \o & \o & \o & \u & \u & \o & \o & \o & \r \\
	\texttt{@inproceedings} & \u & \o & \u & \u & \r & \u & \u & \o & \r & \o & \o & \u & \o & \r & \o & \u & \r & \u & \u & \u & \u & \u & \u & \u & \u & \o & \u & \o & \o & \o & \o & \o & \o & \u & \u & \o & \u & \u & \o & \o & \o & \o & \u & \u & \u & \o & \u & \u & \u & \u & \u & \u & \u & \u & \o & \o & \o & \o & \o & \o & \o & \o & \o & \u & \o & \o & \o & \o & \u & \u & \u & \u & \u & \o & \u & \u & \o & \o & \o & \o & \o & \o & \o & \o & \o & \o & \o & \o & \o & \o & \u & \u & \u & \o & \o & \o & \o & \o & \o & \r & \o & \u & \u & \o & \o & \o & \u & \o & \o & \o & \o & \r \\
	\texttt{@reference} & \u & \o & \o & \u & \u & \u & \u & \u & \u & \u & \o & \o & \o & \r & \o & \o & \r & \o & \u & \o & \u & \o & \u & \u & \u & \o & \u & \o & \o & \o & \u & \u & \o & \u & \o & \o & \u & \u & \o & \o & \o & \o & \u & \o & \u & \o & \u & \u & \u & \u & \u & \u & \u & \u & \o & \o & \o & \o & \o & \o & \o & \o & \u & \u & \o & \o & \o & \u & \u & \o & \u & \u & \u & \o & \o & \u & \o & \o & \o & \o & \o & \o & \o & \o & \o & \o & \o & \o & \o & \o & \u & \u & \u & \o & \o & \o & \o & \o & \o & \r & \o & \o & \u & \o & \o & \u & \u & \o & \o & \o & \o & \r \\
	\texttt{@mvreference} & \u & \o & \o & \u & \u & \u & \u & \u & \u & \u & \u & \o & \o & \r & \o & \o & \r & \o & \u & \o & \u & \o & \u & \u & \u & \o & \u & \o & \o & \o & \u & \u & \o & \u & \o & \o & \u & \u & \o & \o & \o & \o & \u & \o & \u & \o & \u & \u & \u & \u & \u & \u & \u & \u & \o & \o & \o & \o & \o & \u & \u & \u & \u & \u & \o & \o & \o & \u & \u & \o & \u & \u & \u & \u & \o & \u & \u & \o & \o & \o & \o & \o & \o & \o & \o & \o & \o & \o & \o & \o & \u & \u & \u & \o & \o & \o & \o & \o & \o & \r & \o & \o & \u & \o & \o & \u & \u & \u & \o & \o & \o & \r \\
	\texttt{@inreference} & \u & \o & \o & \u & \r & \u & \u & \o & \r & \o & \o & \o & \o & \r & \o & \o & \r & \o & \u & \o & \u & \o & \u & \u & \u & \o & \u & \o & \o & \o & \u & \u & \o & \u & \o & \o & \u & \u & \o & \o & \o & \o & \u & \o & \u & \o & \u & \u & \u & \u & \u & \u & \u & \u & \o & \o & \o & \o & \o & \o & \o & \o & \u & \u & \o & \o & \o & \u & \u & \o & \u & \u & \u & \o & \u & \u & \o & \o & \o & \o & \o & \o & \o & \o & \o & \o & \o & \o & \o & \o & \u & \u & \u & \o & \o & \o & \o & \o & \o & \r & \o & \o & \u & \o & \o & \u & \u & \o & \o & \o & \o & \r \\
	\texttt{@report} & \u & \o & \u & \u & \r & \u & \u & \u & \u & \u & \o & \u & \o & \r & \o & \u & \u & \u & \u & \u & \u & \u & \u & \u & \u & \o & \u & \o & \o & \o & \u & \u & \o & \u & \u & \o & \u & \u & \o & \o & \o & \o & \r & \u & \u & \u & \u & \o & \u & \u & \u & \u & \u & \u & \o & \o & \o & \o & \o & \u & \u & \u & \o & \u & \o & \o & \o & \u & \u & \u & \u & \u & \u & \o & \o & \u & \u & \o & \u & \o & \o & \o & \o & \o & \o & \u & \o & \o & \o & \o & \u & \u & \u & \o & \o & \o & \o & \o & \o & \r & \o & \u & \r & \o & \o & \u & \o & \u & \u & \o & \o & \r \\
	\texttt{@thesis} & \u & \o & \u & \u & \r & \u & \u & \u & \u & \u & \o & \u & \o & \r & \o & \u & \u & \u & \u & \u & \u & \u & \u & \u & \u & \o & \u & \o & \o & \o & \u & \u & \o & \u & \u & \o & \u & \u & \o & \o & \o & \o & \r & \u & \u & \o & \u & \u & \u & \u & \u & \u & \u & \u & \o & \o & \o & \o & \o & \u & \u & \u & \o & \u & \o & \u & \o & \u & \u & \u & \u & \u & \u & \o & \o & \u & \u & \o & \u & \o & \o & \o & \o & \o & \o & \u & \o & \o & \o & \o & \u & \u & \u & \o & \o & \o & \o & \o & \o & \r & \o & \u & \r & \o & \o & \u & \u & \u & \u & \o & \o & \r \\
	\texttt{@unpublished} & \u & \o & \u & \u & \r & \u & \u & \u & \u & \u & \u & \u & \o & \r & \u & \u & \u & \u & \u & \u & \u & \u & \u & \u & \u & \o & \u & \u & \u & \u & \u & \u & \o & \u & \u & \o & \u & \o & \o & \o & \o & \o & \u & \u & \u & \o & \u & \u & \u & \u & \u & \u & \u & \u & \o & \o & \o & \o & \o & \u & \u & \u & \o & \u & \o & \u & \o & \u & \u & \u & \u & \u & \u & \u & \u & \u & \u & \o & \u & \o & \o & \o & \o & \o & \o & \u & \o & \o & \o & \o & \u & \u & \u & \o & \o & \o & \o & \o & \o & \r & \o & \u & \u & \o & \o & \u & \u & \u & \u & \o & \o & \r \\
	\bottomrule
\end{tabular}

		}
		\newlength\tablecoffinwidth
		\setlength\tablecoffinwidth%
			{(\textwidth-\CoffinWidth\TableCoffin)/2}
		\TypesetCoffin\TableCoffin[t, l](\tablecoffinwidth, 0pt)
	\end{minipage}
\end{table}

Även om formatet på databaserna är enkelt att redigera för hand kan det
vara skönt att använda något program som gör det enklare. Några sådana
program, bland annat \emph{JabRef}%
\footnote{\url{http://jabref.sourceforge.net/}} (open-source, finns 
tillgängligt för alla vanliga plattformar) och \emph{BibDesk}%
\footnote{\url{http://bibdesk.sourceforge.net/}} (för Mac OS~X), nämndes
lite kort på~\cpageref{sec:5}. Det finns såklart många fler, både
kommersiella (till exempel \emph{Mendeley}) och open-source-baserade, 
men att lista alla är inte denna bokens syfte.

Många webbaserade sökverktyg för akademiska publikationer, till exempel
Scopus\footnote{\url{http://www.scopus.com/home.url}}, gör det möjligt att
enkelt exportera information om de artiklar man refererar till i det
format \BibTeX{} och \pack{biblatex} använder, och dessa kan sedan
läggas till direkt i den egna referensdatabasen.
Detta är mycket praktiskt i större projekt så som kandidatuppsatser.

\subsection{Stilar och inställningar i \pack{biblatex}}
Nästa steg efter att ha skapat en referensdatabas är att inkludera
paketet \pack{biblatex}. Paketet är gjort för att kunna vara relativt
generellt, och har därför en uppsjö olika inställningar och
bibliografistilar att välja mellan, alla dokumenterade i manualen
\parencite{Lehman13}.

\subsubsection{Vanliga inställningar}
\index{inställningar!biblatex@\pack*{biblatex}}
Eftersom \pack{biblatex} är så omfattande är det omöjligt att gå igenom
alla de inställningar som behandlas av manualen \parencite{Lehman13}. Det
finns dock ett par inställningar som är viktiga att hålla koll på eftersom
de berör saker man ofta vill ändra på.

Den första inställningen man alltid bör använda är \texttt{style}. Denna
inställning berättar vilken bibliografistil \pack{biblatex} ska använda
i dokumentet, där \texttt{numeric} är standardvärdet. Bibliografistilarna
behandlas lite mer utförligt senare i denna del.

Nästa viktiga inställning är \texttt{backend}, som kontrollerar vilket
program \pack{biblatex} ska använda för att processera referenserna
(\eg vilket program som måste köras efter första \pdfLaTeX-rundan).
Standardvärdet är \texttt{biber}, och det finns nästan aldrig någon
anledning att ändra på detta. Det kan dock vara värt att komma ihåg
att just \pack{biber} används. Den intresserade kan även läsa
\pack{biber}-manualen \parencite[\ppno~3\psqq]{Kime13} för lite kort
historik kring \pack{biber} och anledningen till att just \pack{biber}
bör användas.

Utöver dessa finns inställningar för hur referenser sorteras, hur många
namn som visas, vilket datumformat som ska användas och liknande, men
dessa sätts ofta av bibliografistilen.
Det finns även stilspecifika inställningar (\texttt{isbn}, \texttt{url},
\texttt{doi} och \texttt{eprint}) som kontrollerar om några av de
tillhörande fälten ska döljas från referenslistan.

\subsubsection{Bibliografistilar}
\index{bibliografi!-stil|(}
Bibliografistilarna kontrollerar dels utseendet på själva referenslistan
men även hur de kommandon som används för att referera (\cmd{textcite},
\cmd{parencite} och \cmd{footcite}) beter sig.
\Crefrange{fig:bibstyle:1}{fig:bibstyle:2} visar ett antal olika
bibliografistilar, några som följer med \pack{biblatex} och några som
följer med andra paket eller samlingar.

En bibliografistil laddas i \pack{biblatex} genom inställningen
\texttt{style}, och man kan till exempel ladda den alfabetiska stilen
som visas i \cref{fig:bibstyle:alphabetic} på följande sätt:
\latex|\usepackage[style=alphabetic]{biblatex}|

\Cref{fig:bibstyle:1} visar som sagt de bibliografistilar som inkluderas
med \pack{biblatex}. Vanligt inom matematik och fysik är att man använder
något liknande \texttt{numeric} (\cref{fig:bibstyle:numeric}), men på
Chalmers använder man istället författare-årtal
(\cref{fig:bibstyle:authoryear}), eller om möjligt Chicago-stilen
(som visas i \cref{fig:bibstyle:chicago} och implementeras av ett eget
paket, \pack{biblatex-chicago}, som diskuteras senare i detta kapitel).
Många fält och publikationer har egna specifika stilar, varav de flesta
stora publikationer har \pack{biblatex}-stilar — ett urval av sådana
stilar visas i \Cref{fig:bibstyle:2}, men den intresserade hänvisas till
CTAN-katalogen\footnote{\url{http://www.ctan.org/tex-archive/macros/latex/contrib/biblatex-contrib}}, där fler stilar listas.

\subsubsubsection[]{Chicago-stilen med \pack{biblatex-chicago}}
Chicago-stilen \parencite{Chicago10} är väldigt omfattande (även om man
bortser från de delar som inte relaterar till referenslistor), och av
tekniska skäl finns den därför inte implementerad som en enkel
\pack{biblatex}-stil. Istället implementeras stilen som ett paket,
\pack{biblatex-chicago}, som laddar \pack{biblatex} automatiskt.

Paketets manual \parencite{Fussner13} förklarar i stora drag hur det
är tänkt att användas, men i princip gäller samma inställningar som
för \pack{biblatex} bortsett från bibliografistilar. Istället för alla
de stilar som finns tillgängliga för \pack{biblatex} använder sig
\pack{biblatex-chicago} endast av två: fotnoter (\texttt{notes})
och författare-årtal (\texttt{authordate}, den stil som rekommenderas
av Chalmers och som syns i \cref{fig:bibstyle:chicago}).
För att använda paketet istället för någon annan \pack{biblatex}-stil
räcker det oftast med att istället ladda \pack{biblatex-chicago}:
\latex|\usepackage[authordate,backend=biber]{biblatex-chicago}|
\index{bibliografi!-stil|)}

\clearpage
% NOTE: make sure first figure is on even page, second on odd
\begin{figure}[p]
	\centering
	\subfloat[Alfabetisk stil (\texttt{alphabetic})]{
		\label{fig:bibstyle:alphabetic}% (fold)
		\fbox{%
			\includegraphics[width=0.9\textwidth,trim=0 0 0 0]%
				{\latexbokFiguredir/5/alphabetic.pdf}%
		}
	} \\
	\subfloat[Författare-titel-stil (\texttt{authortitle})]{
		\label{fig:bibstyle:authortitle}% (fold)
		\fbox{%
			\includegraphics[width=0.9\textwidth,trim=0 0 0 0]%
				{\latexbokFiguredir/5/authortitle.pdf}%
		}
	} \\
	\subfloat[Författare-årtal-stil (\texttt{authoryear})]{
		\label{fig:bibstyle:authoryear}% (fold)
		\fbox{%
			\includegraphics[width=0.9\textwidth,trim=0 0 0 0]%
				{\latexbokFiguredir/5/authoryear.pdf}%
		}
	} \\
	\subfloat[Numerisk stil (\texttt{numeric})]{
		\label{fig:bibstyle:numeric}% (fold)
		\fbox{%
			\includegraphics[width=0.9\textwidth,trim=0 0 0 0]%
				{\latexbokFiguredir/5/numeric.pdf}%
		}
	}
	\caption{Fyra av de standardstilar som följer med \pack{biblatex}.}
	\label{fig:bibstyle:1}
\end{figure}
\begin{figure}[p]
	\centering
	\subfloat[American Psychological Association-stil (\texttt{apa})]{
		\label{fig:bibstyle:apa}% (fold)
		\fbox{%
			\includegraphics[width=0.9\textwidth,trim=0 0 0 0]%
				{\latexbokFiguredir/5/biblatex-apa.pdf}%
		}
	} \\
	\subfloat[American Chemical Society-stil (\texttt{chem-acs})]{
		\label{fig:bibstyle:chem-acs}% (fold)
		\fbox{%
			\includegraphics[width=0.9\textwidth,trim=0 0 0 0]%
				{\latexbokFiguredir/5/biblatex-chem-acs.pdf}%
		}
	} \\
	\subfloat[Chicago-stil (\pack{biblatex-chicago})]{
		\label{fig:bibstyle:chicago}% (fold)
		\fbox{%
			\includegraphics[width=0.9\textwidth,trim=0 0 0 0]%
				{\latexbokFiguredir/5/biblatex-chicago.pdf}%
		}
	} \\
	\subfloat[Institute of Electrical and Electronics Engineers-stil (\texttt{ieee})]{
		\label{fig:bibstyle:ieee}% (fold)
		\fbox{%
			\includegraphics[width=0.9\textwidth,trim=0 0 0 0]%
				{\latexbokFiguredir/5/biblatex-ieee.pdf}%
		}
	}
	\caption{Några av de bibliografistilar som definieras av olika paket och inte följer med \pack{biblatex}.}
	\label{fig:bibstyle:2}
\end{figure}

\clearpage
\subsection{Referenskommandon och referenslistor}
\index{bibliografi}
När man så skapat sin referensdatabas (som givetvis uppdateras i takt med
att rapporten blir större), valt bibliografistil och laddat
\pack{biblatex} är det dags att börja referera till artiklar.

\Textcite[\ppno~79\psqq]{Lehman13} listar en uppsjö med kommandon för att
åstadkomma detta, men det är egentligen bara fyra eller fem som är viktiga
att ha koll på. Fyra av dessa har dessutom varianter som kan vara bra att
känna till, och alla fem tar i sitt huvudargument en kommaseparerad lista
av referensnycklar från referensdatabasen.

Alla referenskommandon (utom \cmd{nocite}) tar två frivilliga argument
utöver den obligatoriska listan av referensnycklar. Det första frivilliga
argumentet används för att sätta in text före referensen (vilket är
praktiskt speciellt när man använder \cmd{parencite}), och burkar oftast
inte användas alls. Det andra (eller det enda, om bara ett anges) används
för att lägga till text \emph{efter} referensen, vilket ofta används för
att ange sidnummer. Då finns även kommandon \cmd{pno}, \cmd{ppno},
\cmd{psq} och \cmd{psqq} som formaterar sidhänvisningar snyggt, men oftast
ska \pack{biblatex} kunna göra detta automatiskt om argumentet bara består
av siffror:
\latex|\parencite[12-14]{Lehman13}|

\begin{description}
	\item[\cmd{nocite}] används då man vill ha med en referens i 
		referenslistan utan att behöva referera till den i text. Kommandot
		tar även ett specialargument, \texttt{*}, som indikerar att alla
		referenser i databasen ska inkluderas i listan.
	\item[\cmd{textcite}] används då man refererar i löpande text, \eg
		där man använder referensen som ett substantiv
		(\enquote{\ldots{}NN har visat att\ldots}).
	\item[\cmd{parencite}] används i de fall då man vill
		\enquote{indirekt} referera till en källa, exempelvis genom
		att referera till tidigare resultat (\enquote{\ldots{}tidigare
		resultat (NN) har varit positiva\ldots}).
	\item[\cmd{footcite}] används när man vill referera genom en fotnot.
	\item[\cmd{autocite}] försöker vara intelligent och använda det
		kommando som passar bäst i situationen med hänsyn till den
		bibliografistil som används, men bör bara användas i situationer
		där man normalt skulle använda \cmd{parencite} eller
		\cmd{footcite}. Kommandot förklaras tydligare i manualen
		\parencite[\pno~82]{Lehman13}.
\end{description}

De fyra sista har varianter för multipla referenser med olika 
sidhänvisningar, definierade på formen
\latex|\textcites[<sidor>]{<referens 1>}[<sidor>]{<referens 2>}...|
med analoga ekvivalenter för \cmd{parencite} och \cmd{footcite}.
Trion \cmd{textcite}, \cmd{parencite} och \cmd{autocite} (och deras
multipelvarianter) har dessutom varianter à la \cmd{Textcite}, som är
gjorda för att användas i början på en mening eller i andra situationer
där texten bör inledas med en versal.
%
%% NOTE: Reset interaction mode (see top of this file)
\interactionmode=\previousmode
%% END NOTE
\end{document}

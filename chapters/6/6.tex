\ifdefined\latexbokFontdir\else\def\latexbokFontdir{../../fonts}\fi
\ifdefined\latexbokFiguredir\else\def\latexbokFiguredir{../../examples}\fi
\documentclass[lang=sv,ptsize=10pt,font=none,nomath,titles=bf,../../a4.tex]{subfiles}
\begin{document}
\section{Vidare läsning}\label{sec:6}
Den här introduktionen har förhoppningsvis gett dig en bra \LaTeX-grund
som låter dig typsätta både rapporter och artiklar utan problem. Tyvärr
kommer du, eftersom \LaTeX{} är ett så stort system, sannolikt att behöva
ytterligare hjälp, tips och resurser allt eftersom du använder \LaTeX{} 
och stöter på problem eller svårigheter.

Denna del av introduktionen ämnar ge dig några tips på sådana resurser.
Delen inleds med några relevanta resurser så som böcker och journaler
som behandlar \TeX{} och \LaTeX{} på en mer eller mindre avancerad nivå,
samt några tips när det gäller att få hjälp med specifika problem.
Därefter följer några tips gällande stora projekt, som du sannolikt kommer
att ha nytta av förr eller senare, speciellt om du studerar på en högskola
eller ämnar författa böcker och liknande.

Därefter följer en lista av viktiga och nyttiga \LaTeX-paket, som man bör
åtminstone skumma igenom för att få en hyfsad uppfattning om vilka paket
som finns och när man bör använda dem. Avslutningsvis ges även en lista
över andra projekt som liknar \LaTeX, till exempel det tidigare nämnda
\XeTeX.

\subsection{Andra resurser}
Även om denna introduktion siktar på att ge dig allt du behöver för att
lära dig \LaTeX{} är det inte säkert att den är tillräcklig. Man kan inte
diskutera allt i en kort introduktion, och vill man lära sig mer om den
inre strukturen hos \TeX{} eller \LaTeX{} så finns det redan mycket bra 
och utförliga resurser tillgängliga.
Dessutom kan man i en kort introduktion inte diskutera specifika problem
i detalj, eftersom dessa ofta beror på den specifika situationen.
Nedan följer därför en lista över resurser i form av böcker, journaler
och forum som kan hjälpa dig släcka din kunskapstörst eller lösa dina
\LaTeX-problem.

\subsubsection{Böcker och artiklar}
Det finns mycket material tillgängligt när det gäller \LaTeX{}. Många
böcker inom ämnet har publicerats av Addison-Wesley, och det finns även
en uppsjö artiklar, böcker och guider tillgängliga på CTAN, och en journal
(\emph{The Prac\TeX{} Journal}\footnote{\url{http://tug.org/pracjourn/}})
som finns
tillgänglig på internet. Nedan följer en lista på några av de böcker och 
artiklar som kan vara av intresse för dig som precis börjat med \LaTeX.

\begin{description}
	\item[\emph{An essential guide to \LaTeXe{} usage} \parencite{Fenn07}]
	brukar också refereras till som \pack{l2tabu} (dess namn på CTAN), och
	ger en kort lista över utdaterade paket, dödssynder inom \LaTeX{} samt
	några små tips. Förhoppningsvis lär du dig inget nytt av att läsa
	\pack{l2tabu} (det skulle ju indikera att den här introduktionen är
	felaktig), men det kan vara värt att skumma igenom den ändå.
	
	\item[\emph{The Not So Short Introduction to \LaTeXe} \parencite{Oetiker11}]
	går även under namnet \pack{lshort} och är
	en kort introduktion till \LaTeX,
	skriven på engelska. Den går \emph{lite} djupare in på vissa bitar
	av \LaTeX{}, och fokuserar en del på den nu i princip överflödiga
	\LaTeX-motorn som skapar \DVI-filer, men har varit en stor inspiration
	till den här introduktionen. Absolut värd en (snabb) genomläsning.
	
	\item[\emph{Math into \LaTeX} \parencite{Gratzer96}]
	ger än ännu längre introduktion till \LaTeX{} än \pack{lshort}, och
	är även den skriven på engelska. En bra ytterligare referens om det är
	något man vill veta mer om eller något man tycker är otydligt i den
	här introduktionen och \pack{lshort}.
	
	\item[\emph{Math Mode} \parencite{Voss10}]
	ger en mycket ordentlig
	genomgång av matematiktypsättning både i vanliga \LaTeX{} och med
	\AmS\LaTeX{}, och är en nästintill oumbärlig referens när man skriver
	lite mer komplicerade ekvationer. Bokmärk och titta igenom varje gång
	du skriver matematik i \LaTeX.
	
	\item[\emph{Short Math Guide for \LaTeX} \parencite{Downes02}]
	är en kort guide till matematik med \AmS\LaTeX{} skriven av en av
	huvudpersonerna bakom paketsamlingen. Ger några små värdefulla tips,
	en lista över symboler och en introduktion till
	\cmd{DeclareMathOperator} och några andra \AmS-konstruktioner som inte
	diskuteras utförligt i den här introduktionen. Sjutton mycket läsvärda
	sidor.
	
	\item[\emph{The \LaTeX{} Companion} \parencite{Mittelbach04}]
	beskriver alla \LaTeX-\\kommandon och en stor mängd paket. Om du ska
	skriva ett eget paket eller en egen dokumentklass, eller bara är
	intresserad av att ”hacka” \LaTeX{} lite, så bör du absolut ta en titt
	på den här boken. En mycket bra referens för den vane \TeX{}aren.
	
	\item[\emph{\LaTeX: A Document Preparation System} \parencite{Lamport94}]
	skrevs av\\ författaren till \LaTeX{} och kan anses vara det närmsta en
	”manual” till \LaTeX{} man kan komma. Främst av historiskt intresse,
	och inget för nybörjaren.
	
	\item[\emph{\TeX{} by Topic} \parencite{Eijkhout92}]
	ger en utförlig förklaring till \TeX{} och är en relevant referens för
	alla som ska utföra lågnivåarbete i \TeX{} eller \LaTeX{}. Främst
	riktad till mycket vana användare av \TeX{} och \LaTeX{}, och absolut
	inte riktad till nybörjaren.
	
	\item[\emph{The \TeX{}book} \parencite{Knuth86}]
	skrevs av skaparen av \TeX{} och förklarar i detalj hur systemet
	fungerar. Främst av historiskt intresse, och egentligen bara värd att
	titta på om man vill veta hur \TeX{} \emph{egentligen} fungerar.
\end{description}

Utöver dessa bör man givetvis även läsa manualerna till de paket man
använder (dessa finns alltid på CTAN) och kanske även manualen till
\BibTeX{} \parencite{Patashnik88a} om man behöver det.

\subsubsection{Hjälp med specifika problem}
Det tar lång tid att bemästra \LaTeX{} fullt ut, och i början kommer man
garanterat att stöta på problem. Några av dem kanske går att lösa med den
hjälp som ges i den här introduktionen och de andra böcker och artiklar
som presenteras, men vissa måste man fråga någon om.

Det finns så klart en uppsjö olika maillistor, forum och sökmotorer (och
phaddrar, för dig som går på en teknisk högskola) som kan användas för att
lösa problem, ställa frågor och utforska \LaTeX. En mycket bra resurs är
\emph{\TeX{} Stack Exchange}\footnote{\url{http://tex.stackexchange.com/}}
där man kan ställa frågor om \TeX, \LaTeX{} och andra relaterade system.
Många stora namn i \TeX-världen dyker upp där lite då och då, och undrar
man något om \LaTeX{} så är det ett utmärkt ställe att fråga. Använder man
IRC kan man också med fördel besöka \verb|#latex|-kanalen på Freenode%
\footnote{\url{irc://chat.freenode.net}}.

\phantomsection
\subsection{Tips för stora projekt}
För stora \LaTeX-projekt (till exempel kandidatarbeten, examensarbeten
och liknande) är det viktigt att kunna ha en ordentlig ordning på sitt
dokument. Är det dessutom ett dokument många ska samarbeta med är det
ännu viktigare att det är strukturerat.

Den första tumregeln, som även bör tillämpas vid mindre projekt, är att
man bör lägga varje \LaTeX-dokument (eller projekt) i en egen undermapp.
Ännu bättre blir det om man även lägger alla externa filer (figurer,
inkluderad programkod och så vidare) i ännu en undermapp. En bra
mappstruktur skulle alltså kunna se ut ungefär som i \cref{ex:mapp}.

\begin{kod}
	\begin{textcode}
.
`-- FFM233-projekt
|-- img
|   |-- degradering-1.png
|   |-- degradering-2.png
|   |-- ...
|   |-- triangel-1.png
|   `-- triangel-2.png
|-- kod
|   |-- diffekv.m
|   |-- ode.m
|   `-- plot.m
|-- projekt.tex
`-- referenser.bib
	\end{textcode}
	\caption{En bra mappstruktur för ett enkelt \LaTeX-projekt.}
	\label{ex:mapp}
\end{kod}

Utöver detta kan man se till att försöka abstrahera bort till exempel
kommandodefinitioner eller stiländringar till ett eget paket eller en
egen dokumentklass, om möjligt. Att göra detta lämnas som en övning åt
läsaren, men mer information om hur man gör sådant ges av bland annat
\textcites{Flynn06}{LaTeX3}{Robertson06}.

\subsubsection{Versionshantering}
För större projekt och projekt som utförs i grupp är det mycket praktiskt
att kunna spåra ändringar och ändra dokumentet från många olika platser
(gärna samtidigt). Med hjälp av ett versionshanteringssystem så som 
Subversion\footnote{\url{http://subversion.tigris.org/}} blir detta
enkelt. Ännu bättre blir det om man använder ett distribuerat system så
som Mercurial\footnote{\url{http://mercurial.selenic.com/}} eller
Git\footnote{\url{http://git-scm.com/}}, hostat helt gratis av något snällt
företag, till exempel på Bitbucket\footnote{\url{https://bitbucket.org/}}
(Github\footnote{\url{https://github.com/}} eller Gitorious%
\footnote{\url{http://gitorious.org/}} om man föredrar Git framför Mercurial).

Att utförligt förklara hur Mercurial fungerar är utanför den här bokens
område, men är man intresserad bör man läsa
\emph{Hg Init}\footnote{\url{http://hginit.com/}}, som trots sitt bruk av
Comic Sans är en utmärkt resurs för den som vill lära sig Mercurial och
versionshantering.

\subsubsection{Uppdelning av dokumentet}
Stora projekt (med flera kapitel eller stora delar) kan med fördel delas
upp på flera filer. Man har då en grundfil (säg, \texttt{projekt.tex}) som
refererar till flera olika underfiler (kanske en per kapitel, lämpligen
placerade i en undermapp till projektet). Dessa kan man sedan inkludera
i grundfilen på ett antal olika sätt, varpå man bara kompilerar

Den första metoden för att inkludera underfilerna är \cmd{input}, som i
princip lägger in koden från underfilen direkt där kommandot används och
fortsätter kompilera som om koden alltid funnits där. Detta är praktiskt
om man av någon anledning inte vill använda den andra metoden.

Den andra metoden, \cmd{include}, är i princip ekvivalent till \cmd{input}
med den skillnaden att en sidbrytning läggs in före och efter den 
inkluderade koden. Dessutom kan man med hjälp av kommandot
\cmd{includeonly} bestämma \emph{vilka} underfiler som inkluderas, för att
till exempel spara tid vid kompileringen om man bara är intresserad av en
specifik del.

Den tredje och sista metoden är med paketet \pack{subfiles}. Denna är
generellt sett att föredra om den är tillgänglig, och fungerar ungefär som
metod ett men med undantaget att man även kan kompilera underfilerna
var för sig, om man är intresserad av detta. Det kan vara lämpligt att
göra om man bara redigerar ett kapitel, och inte vill kompilera hela
projektet under tiden. Paketet \pack{subfiles} finns på CTAN men inte i
Chalmers datorsystem eller i \TeX{} Live.

\subsection{Rekommenderade paket}
Eftersom \TeX{} är Turingkomplett kan i princip allt göras med språket,
även om det främst är tänkt för typsättning. Således kan det mesta i 
typsättningsväg lösas relativt enkelt. Kan det inte det, och problemet
man vill lösa är ett relativt vanligt problem, så finns det sannolikt ett
paket som löser problemet. Dessa paket finns listade på CTAN (se
\cpageref{sec:ctan}) och kan, om man använder \TeX{} Live, installeras
med verktyget \texttt{tlmgr}.

Dokumentation för alla paket finns på CTAN och kan
även visas genom att skriva \texttt{texdoc \emph{paketnamn}} i terminalen,
eller på
\href{http://texdoc.net}{\nolinkurl{http://texdoc.net/pkg/<paketnamn>}}.

\subsubsection{Allmänt nyttiga paket}
Dessa paket finns i princip alltid och tillhandahåller så grundläggande
funktionalitet att man alltid bör använda dem. Det handlar om allt från
tecken- och typsnittskodning till avstavning och interna länkar.

\begin{description}
	\item[\pack{babel}] (ej med \XeTeX)
	översätter interna strängar (till exempel ”Referenser” och 
	”Sammanfattning”) till det språk som önskas och gör det möjligt för
	dessa språk att avstavas ordentligt. Paketet är ett måste då man
	använder \LaTeX, men har ersatts av \pack{polyglossia} i \XeTeX.
	
	\item[\pack{fixltx2e}]
	korrigerar en del buggar i \LaTeXe-kärnan och gör till 
	exempel matematikkommandon robusta. Inkludera alltid.
	
	\item[\pack{fontenc}] (ej med \XeTeX och \hologo{LuaTeX})
	låter användaren välja typsnittskodning. Från början använder \LaTeX{}
	\texttt{OT1}, som inte innehåller icke-anglikanska tecken så som
	\emph{å}, \emph{ä} och \emph{ö}. Detta medför problem när det gäller
	bland annat avstavning, och man bör därför istället använda 
	\texttt{T1}. Detta diskuteras närmre på \cpageref{pack:fontenc}.

	\item[\pack{fontspec}] (endast med \XeTeX och \hologo{LuaTeX})
	låter användaren välja \textsc{OTF}-typsnitt helt fritt, antingen
	från typsnitt installerade på datorn eller från \textsc{TTF}- eller
	\textsc{OTF}-filer.
	
	\item[\pack{hyperref}]
	gör om innehållsförteckningar, referenser och URIer till riktiga
	länkar i de fall detta stöds (det vill säga om slutformatet är \PDF).
	Dessutom skapar den inbyggda innehållsförteckningar som kan användas
	för att navigera i dokumentet i en del \PDF-läsare (till exempel Skim
	och Acrobat Reader).
	Ett oumbärligt paket som alltid bör inkluderas.
	
	\item[\pack{inputenx}] (ej med \XeTeX och \hologo{LuaTeX})
	är ett paket som låter användaren berätta för \LaTeX{} vilken
	teckenkodning indatan sparats med. De vanigaste inställningarna är
	\texttt{utf8} och \texttt{latin1}, men många andra teckenkodningar
	stöds också.
	Detta paket introducerades lite kort på \cpageref{pack:inputenx}.

	\item[\pack{kpfonts}]
	inkluderar typsnittsfamiljen \emph{Kepler} som är en gratis familj
	baserad på Adobe Palatino. Personligen tycker jag att \emph{Kepler}
	ser bättre och tydligare ut än \emph{Latin Modern}, och paketet
	innehåller dessutom en enorm mängd typsnitt som täcker allasituationer.
	Välj antingen detta paket \emph{eller} \pack{lmodern}, inte båda.
	
	\item[\pack{lmodern}]
	inkluderar typsnittet \emph{Latin Modern} för att ersätta
	\emph{Computer Modern}, standardtypsnittet i \LaTeX. Latin Modern ser
	i princip likadant ut som Computer Modern, men är skarpare och har
	fler glyfer. Det här paketet motiveras lite närmre på 
	\cpageref{sec:2:lmodern}. Välj antingen detta paket \emph{eller}
	\pack{kpfonts}, inte båda.
	
	\item[{\pack{nag}}]
	försöker upptäcka saker som är \emph{bad practice}, saker som kanske
	inte fungerar som man tänkt sig, och liknande. Bör inkluderas med
	alternativen \texttt{l2tabu} och \texttt{orthodox}:
	\latex|\usepackage[l2tabu,orthodox]{nag}|

	\item[{\pack{polyglossia}}] (endast med \XeTeX)
	ersätter \pack{babel} för \XeTeX och har bättre stöd för \UTF.
\end{description}

\subsubsection{Paket för snyggare typsättning}
Även om \LaTeX{} är mycket bra på att typsätta så blir det ibland inte
så snyggt som man kanske kan önska. Vid dessa tillfällen kan man
applicera diverse olika paket för att få hjälp med detta. Det gäller
till exempel tabeller, figurtexter, SI-enheter och matematik.

\begin{description}
	\item[\pack{booktabs}]
	är ett paket som gör det möjligt att skapa mycket snygga tabeller.
	Paketet diskuteras redan på \cpageref{pack:booktabs}, och är i princip
	ett måste om man ska inkludera tabeller i sitt \LaTeX-dokument.

	\item[\pack{caption}]
	gör det möjligt att ändra stilen på figur- och tabelltexter så att de
	syns tydligare och inte smälter in i texten.

	\item[\pack{csquotes}]
	gör det lättare att typsätta citat på ett snyggt och korrekt sätt.
	Bland annat definieras kommandot \cmd{enquote}, som kan användas
	istället för de (inkorrekta) citattecknen \verb|``| och \verb|''|, 
	och kommandot \cmd{blockquote} för längre block av citerad text.

	\item[\pack{fnpct}]
	ser till att fotnoter nära skiljetecken typsätts på ett snyggt sätt,
	och gör det lättare att typsätta flera fotnoter brevid varandra.
	Introducerades kort på \cpageref{sec:2:footnote}.
	
	\item[{\pack{siunitx}}]
	gör det enklare att typsätta SI-enheter, decimaltal och vinklar på ett
	korrekt sätt för icke engelskspråkiga rapporter. Mycket användbart
	paket som tyvärr inte finns på Chalmers datorer. Förklaras lite kort
	på \cpageref{sec:3:siunitx}.
\end{description}

\subsubsubsection*{Paket för matematiktypsättning}
Som teknisk matematiker (eller fysiker) kommer stora delar av de rapporter
man skriver oundvikligen innehålla matematik. Även om \LaTeX{} för sig
självt är relativt bra på att typsätta matematik kan det ibland behövas
några tillägg för att göra det enklare. Många av dessa tillhandahålls av
\AmS{} i det som kallas \AmS\LaTeX, men det finns fler relevanta
matematikpaket.

\begin{description}
	\item[\pack{amsmath}]
	gås igenom i \cref{sec:3} och är i princip oumbärligt om man ska
	typsätta matematik med \LaTeX. Inkludera alltid detta paket om du
	skriver något som kan tänkas innefatta ekvationer.

	\item[\pack{amssymb}]
	definierar en hel del matematiska symboler, till exempel 
	\cmd{therefore} (\(\therefore\)) och \cmd{ggg} (\(\ggg\)), och 
	kommandot \cmd{mathbb} som ger krittavletecken (som
	används för de grundläggande talmängderna).
	
	\item[\pack{amsthm}]
	definierar omgivningar för att typsätta teorem, satser, lemman, bevis
	och dylikt. Användbart i vissa sammanhang, främst för att typsätta
	föreläsningsanteckningar eller uppgifter. 
	
	\item[{\pack{isomath}}]
	korrigerar typsättningen av grekiska bokstäver, så att även dessa typsätts
	kursivt (som variabler). Introducerar även semantiska kommandon som
	\cmd{vectorsym} och \cmd{tensorsym} för typsättning av vektorer
	och tensorer.
	
	\item[\pack{mathtools}]
	lagar några småfel i \pack{amsmath} och definierar kommandon som till
	exempel \cmd{DeclarePairedDelimiter}, som introducerades på
	\cpageref{cmd:declarepaireddelimiter}.

	\item[\pack{skmath}]
	definierar en del kommandon som gör det lättare att typsätta matematik
	(till exempel \cmd{d} som presenterades på \cpageref{sec:3:integ:kod})
	och omdefinierar andra för att förbättra utseendet.

	\item[{\pack{unicode-math}}]
		(endast med \XeTeX och \hologo{LuaTeX})
	kan användas för att ladda \textsc{OTF}-typsnitt för matematik och
	gör det även möjligt att skriva matematik med Unicode-tecken.

	\item[{\pack{xfrac}}]
	definierar ett kommando \cmd{sfrac}, som låter dig typsätta bråk
	med ett snedstreck: \(\sfrac{2}{3}\).
\end{description}

\subsubsection{Paket för grafik}
Grafik i form av figurer eller illustrationer är givetvis en viktig del
av många rapporter, vare sig det är figurer av uppställningar, plottar
eller flödesscheman. Det finns tre relevanta paket när det gäller grafik
i \LaTeX; ett som används för att importera grafik från externa filer och
två som används för att rita direkt med \LaTeX-kommandon.

\begin{description}
	\item[\pack{graphicx}]
	förklaras kort i \cref{sec:4} och gör det möjligt att inkludera
	figurer från externa filer i sitt \LaTeX-dokument. Mycket användbart
	om man har data från till exempel MATLAB eller Mathematica.
	
	\item[\pack{tikz}]
	nämns också i \cref{sec:4} och gör det möjligt att rita
	vektorbaserade figurer direkt i \LaTeX. Mycket användbart om man
	vill rita exempelvis uppställningar, enklare illustrationer eller 
	flödesscheman, men kan användas till otroligt mycket mer. Fungerar
	endast med \pdfLaTeX{} i \PDF-läge eller moderna varianter som
	\XeTeX.

	\item[\pack{pstricks}]
	är en slags motsvarighet till \pack{tikz} som endast fungerar med de
	\LaTeX-varianter som genererar \DVI- eller \textsc{PS}-filer. Inte
	lika användbart som \pack{tikz} eftersom det inte fungerar med 
	\pdfLaTeX{} i \PDF-läge.
\end{description}

\subsubsection{Paket för bibliografier}
Även om \BibTeX ofta är tillräckligt för att kunna typsätta dokument som
ska skickas till journaler eller förläggare (eftersom verktyget
inkluderar många vanliga engelskspråkiga bibliografistilar) så är det inte
alltid tillräckligt. Till exempel så rekommenderar Chalmers bibliotek
en bibliografistil som inte finns med i \BibTeX och som till råga på allt
ska vara på svenska. Sådana problem löses enkelt av diverse paket.

\begin{description}
	\item[{\pack{chscite}}]
	är ett paket som skapats för att implementera de rekommendationer
	Chalmers Bibliotek publicerat \parencite{ChsLib10} i \BibTeX.
	Rekommenderas absolut för alla sorters rapporter, artiklar och
	publikationer du kan tänkas producera under din studietid, så länge
	det inte finns krav på stil från annat håll. Diskuteras även på
	\cpageref{sec:chscite}.
\end{description}

\subsubsection{Paket som löser problem}
Ibland vill man göra något som är väldigt svårt att göra med vanlig
\LaTeX-kod, till exempel ändra sidstorlek eller skapa underfigurer. Då
många sådana problem är vanligt förekommande finns det ofta paket som
löser dem.

\begin{description}
	\item[\pack{acro}]
	låter dig skapa listor över förkortningar på ett enkelt sätt, men är
	lite enklare än \pack{glossaries} som egentligen är gjort för hela
	ordlistor. Till skillnad från \pack{glossaries} kräver \pack{acro}
	inget externt program. Diskuteras kort på \cpageref{sec:2:acro}.

	\item[\pack{cleveref}]
	definierar kommandot \cmd{cref} (och många liknande kommandon), ett
	alternativ till \cmd{ref} som automatiskt skriver ut vilken typ av
	referens det handlar om. Fullt kompatibelt med \pack{varioref}, och
	tillsammans gör det korsreferenser mycket enklare.

	\item[\pack{enumitem}]
	gör det möjligt att definiera egna sorters listor, och modifierar 
	\env{enumerate}	så att man även kan specificera en egen etikett, återuppta
	numreringen från förra listan, och mycket mer.
	
	\item[\pack{float}]
	definierar ett kommando \cmd{newfloat} som skapar nya sorters flytande
	objekt (vilka diskuterats på \cpageref{sec:floats}).
	
	\item[\pack{geometry}]
	kan användas för att ändra en del mått i \LaTeX{} (till exempel sidans
	storlek eller marginalerna) om så önskas. Oftast behöver man inte göra
	detta, eftersom dokumentklassen har definierat de mått den har av en
	anledning. Kan vara värdefullt om man vill skapa dokument för A5-%
	eller A6-papper (eller andra ovanliga storlekar).

	\item[\pack{glossaries}]
	diskuterades tidigare på \cpageref{sec:2:glossaries}, och låter
	dig skapa ordlistor och listor över förkortningar på ett enkelt
	och relativt automatiserat sätt. Paketet är mycket användbart i
	tekniska dokument där en ordlista är viktig för förståelsen.

	\item[\pack{imakeidx}]
	nämndes på \cpageref{sec:2:imakeidx} och låter dig skapa sakregister
	på ett automatiserat sätt. Paketet kan vara användbart i större
	dokument, främst böcker och liknande facklitteratur.

 	\item[\pack{multicol}]
 	löser problemet med att typsätta text i kolumner på ett mycket bättre
 	sätt än standardklasserna, med hjälp av omgivningen \env{multicols}.
 	Paketet stödjer bland annat korrekt balanserade kolumner.
 	
	\item[\pack{multirow}]
	gör det möjligt att skapa tabellceller som spänner över flera
	rader.
	
	\item[\pack{sidecap}]
	definierar omgivningen \env{SCfigure} som typsätter en figur med
	figurtexten brevid istället för under eller över figuren. Kan vara
	användbart om man har smala figurer med lång figurtext.
	
	\item[\pack{subfig}]
	definierar kommandon som möjliggör skapandet av ”underfigurer”, det
	vill säga grupper av relaterade figurer som alla kan refereras till
	antingen som grupp (till exempel ”Figur 1”) eller individuellt (till
	exempel ”Figur 1a”).
	
	\item[\pack{varioref}]
	gör det möjligt att, med hjälp av kommandot \cmd{vref},
	skapa korsreferenser
	som inte bara refererar till etiketterna med nummer utan även till den
	sida figuren eller tabellen finns på. Detta görs på ett intelligent
	sätt, så att referensen blir ”figur X på nästa sida” om figuren är på
	nästa sida, och så vidare. Mycket relevant paket, speciellt i större
	rapporter.
	
	\item[\pack{wrapfig}]
	skapar ett nytt flytande objekt \texttt{wrapfig} som placerar figuren
	till höger eller vänster på sidan och låter övrig text ”flyta” runt
	figuren.
	
	\item[\pack{xspace}]
	definierar ett kommando \cmd{xspace} som kan läggas till på slutet av
	kommandodefinitioner för att de ska bete sig snällare i brödtext, så
	att man slipper skriva \texttt{\{\}} efter kommandot.
\end{description}

\subsubsection{Paket för att ändra utseende}
Standardklasserna lämnar ofta en del att önska i termer av utseende. Vill
man göra något åt detta kan man använda paket som utformats för att låta
dig ändra stilen av vissa element, till exempel rubriker eller sidhuvuden%
\footnote{Vill man lösa problemet på riktigt ska man givetvis använda en
dokumentklass som ser bättre ut istället.}.

\begin{description}
	\item[\pack{fancyhdr}]
	låter dig förändra sidhuvud och sidfot för att införa snyggare, mer
	informativa eller tydligare stilar. Till exempel så kan texten i
	sidhuvudet ändras för att visa sidnummer och kapitel, medan sidfoten
	ändras för att visa till exempel kontakt- eller copyrightinformation.
	
	\item[\pack{titlesec}]
	gör det möjligt att ändra stilen på rubrikerna i dokumentet, till
	exempel genom att byta textstil eller sättet rubriken typsätts.

	\item[\pack{tocloft}]
	definierar kommandon för att styra utseendet av
	innehållsförteckningen.
\end{description}

\subsubsection{Andra specialiserade paket}
För en del snäva områden så som kvantfysik eller datavetenskap finns det
specialiserade paket som löser specifika uppgifter. Dessa kan med fördel
användas om man skriver rapporter inom området, eller rör vid ämnet i
någon inlämningsuppgift.

\begin{description}
	\item[\pack{algorithmic}]
	är ett paket för typsättning av algoritmer och pseudokod.
	
	\item[{\pack{braket}}]\label{pack:braket}
	definierar kommandon för att handskas med braket-notationen som
	används inom kvantfysik. Kommandot \cmd{Braket} kan således resultera
	i följande ekvation:
	\begin{equation*}
		\Braket{\phi|\dfrac{\partial^2}{\partial t^2}|\psi}
	\end{equation*}
	
	Paketet definierar även kommandot \cmd{Set}, för att på liknande
	sätt typsätta mängddefinitioner:
	\begin{equation*}
		\Set{x\in\mathbf{R}^2|0<{|x|}<5}
	\end{equation*}
	
	\item[\pack{listings}]
	kan användas för att inkludera programkod i \LaTeX-dokument. Det finns
	möjlighet att skriva ut radnummer, lägga till ramar, associera
	figurtexter och även en mycket enkel syntaxfärgning \eng{syntax
	highlighting}. Fungerar inte om koden innehåller svenska tecken, även
	om paketet \emph{\pack{listingsutf8}} försöker lösa detta. Använder 
	man \XeTeX uppstår inte detta problem.
	
	\item[{\pack{minted}}]
	kan sägas vara en förbättring av \pack{listings} som använder Python-%
	programmet \emph{Pygments} för att även sätta färg på koden. Ett bra
	alternativ om man ska inkludera kod och vill ha fullständig
	syntaxfärgning. Paketet lider dock av samma problem med svenska tecken
	som \pack{listings}.
	
	\item[{\pack{todonotes}}]
	låter dig infoga (synliga) ”att göra”-anteckningar i ditt dokument, tillsammans
	med en lista över dessa. Paketet innehåller även ett kommando \cmd{missingfigure},
	som kan användas för att markera en saknad figur.
\end{description}

\subsection{Andra \TeX-baserade projekt}
\LaTeX{} är inte det enda användbara \TeX-baserade projektet, även om det
är det makropaket som används mest i praktiken. Det finns alternativ
som baseras på \LaTeX{} men löser problem (till exempel \XeTeX och
\hologo{LuaTeX}) men även alternativa makropaket som är mycket olika
\LaTeX{} i sin struktur (till exempel \hologo{ConTeXt}). Dessutom finns
det projekt för att utveckla \LaTeXe{} och göra vissa saker som till 
exempel utveckling av dokumentklasser mycket enklare.

\subsubsection{Unicode-baserade \XeTeX}
\XeTeX är, precis som \pdfLaTeX, en \TeX-kompilator. Denna kan användas
på samma sätt som vanliga \TeX{} och \pdfLaTeX (dvs. den kan köras på i
princip samma kod) med hjälp av dess program, \texttt{xelatex}.

Skillnaden mellan \pdfLaTeX{} och \XeTeX som gör att man kanske föredrar
det senare är relativt stor, sett till de djupa delarna av \TeX: \XeTeX 
förutsätter, till skillnad från \pdfLaTeX, att all indata är \UTF och kan
därmed läsa \UTF-dokument på ett korrekt sätt. Detta innebär bland annat
att saker som inte fungerar i \pdfLaTeX ens med \pack{inputenx}, till
exempel svenska tecken i kodlistingar eller inkluderade filer, kommer att
fungera utmärkt med \XeTeX.

Dessutom kan man med hjälp av speciella kommandon välja typsnitt på ett
mycket enklare sätt, eftersom \XeTeX stödjer \textsc{OTF}- och
\textsc{TTF}-typsnitt, och kan hitta alla typsnitt som är installerade
på datorn, inte bara de som finns i form av \LaTeX-paket.

En relativt övergripande bild av vad som är nytt i \XeTeX ges av
\textcite{Robertson11}; för det mesta kan man förutsätta att \LaTeX-kod
kommer att fungera precis likadant i \XeTeX med några få undantag:
\begin{itemize}
	\item Paketet \pack{polyglossia} bör användas istället för \pack{babel}
	\item Varken \pack{inputenx} eller \pack{fontenc} bör användas
	\item Paketet \pack{fontspec} bör användas för att ladda typsnitt
	\item Paketet \pack{unicode-math} kan användas för att ladda typsnitt
		  för matematik och för att skriva matematik med Unicode-tecken.
\end{itemize}

\XeTeX är idag så stabilt att det kan användas istället för \pdfLaTeX.
Den här boken är typsatt med \XeTeX, och eftersom skillnaderna främst
ligger i att \pack{inputenx} och \pack{fontenc} inte bör användas (\XeTeX
klarar nämligen även gamla \pdfLaTeX-typsnitt) kan man i princip alltid
använda \XeTeX istället för \pdfLaTeX.

\subsubsection{Skripta med \hologo{LuaTeX}}
\Hologo{LuaTeX} är en kombination av programmeringsspråket Lua och
typsättningsspråket \TeX. Introduktionen till \hologo{LuaLaTeX},
som är \hologo{LuaTeX}s motsvarighet till vanliga \LaTeX,
beskriver systemet som uppföljaren till \pdfLaTeX:
\begin{quote}
	\begin{english}
		It is the designated successor of pdf\TeX{} and includes all of 
		its core features: direct generation of \PDF files with support 
		for advanced \PDF features and micro-typographic enhancements to 
		\TeX{} typographic algorithms.

		\nopagebreak
		\hfill\textcite{Gonnard10}\hspace{-1ex}%
	\end{english}
\end{quote}

I många avseenden är \hologo{LuaTeX} mycket likt \XeTeX; man kan använda
konventionella typsnitt med båda motorerna (med paketet \pack{fontspec}), 
och de hanterar båda \UTF 
mycket bättre än sina företrädare. Skillnaden är att \hologo{LuaTeX}
även gör det möjligt att bädda in Lua-kod direkt i dokumentet, vilket gör
en del saker mycket enklare eftersom man har tillgång till ett fullgott
skriptspråk. I likhet med \XeTeX är \hologo{LuaTeX} stabilt nog för att
användas istället för \pdfLaTeX.

\subsubsection{Ett alternativ: \hologo{ConTeXt}}
\Hologo{ConTeXt} är ett alternativt makropaket för \TeX{} som utvecklats
parallellt med \LaTeX{} men med en annan inriktning. Medan \LaTeX{}
försöker isolera användaren från typografiska beslut, vilket gör det
lämpligt för att till exempel skicka in artiklar till förlag, så
försöker \hologo{ConTeXt} ge användaren en lite mer strukturerad tillgång
till de typografiska funktionerna i \TeX.

Det finns en del artiklar som jämför \hologo{ConTeXt} och \LaTeX{}
\parencite[till exempel][]{Hoekwater98}, men kontentan av det hela är att
\LaTeX{} är vanligare i akademiska kretsar och bättre på att typsätta
matematik, och att man därför oftast bör hålla sig till \LaTeX. Är man
trots detta intresserad av \hologo{ConTeXt} bör man referera till
\hologo{ConTeXt}-manualen \parencite{Hagen01}.

\subsubsection{Framtiden: \LaTeX3}
Den variant av \LaTeX{} som diskuterats i boken, \LaTeXe{}, härstammar
från 1994 då den ersatte den tidigare versionen, \LaTeX~2.09. Denna
variant uppdateras fortfarande regelbundet, men redan 1997 startade ett
initiativ för att utveckla nästa version av \LaTeX, kallad \LaTeX3
\parencite{Rowley99}, som fortgår än idag. Än så länge är \LaTeX3 inte
något man som användare behöver ha stenkoll på, eftersom det främst
består av en uppsättning paket som underlättar i skrivandet av paket.

Hur \LaTeX3 kommer se ut är inte helt klart, men att det kommer kunna
användas med samma kompilatorer som används idag är givet. Utöver detta
listar \textcite[4\psq]{Rowley99} ett antal mål för projektet.

I dagsläget består projektet som sagt av ett ramverk som underlättar
skrivandet av nya paket. Detta består av en uppsättning konventioner
\parencite{expl3}, som bland annat beskriver hur man ska namnge de
interna kommandon man definierar i ett paket, och en samling paket
\parencite{interface3} som enligt dessa konventioner definierar
kommandon som abstraherar användbara programmeringskoncept (till
exempel listor, Booleska variabler, heltal, flyttal m.fl.).
Av den anledningen är \LaTeX3 något att hålla ögonen på om man är
intresserad av att skriva egna paket eller dokumentklasser, men för
den \enquote{vanliga} användaren ligger \LaTeX3 fortfarande bortom
horisonten.

\end{document}

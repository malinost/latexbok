\RequirePackage[draft]{ifdraft}
\RequirePackage{xstring,xparse}
\NewDocumentCommand\paper{mm}{\IfStrEq{\papersize}{#1}{#2}{}}
\PassOptionsToPackage{usenames,dvipsnames}{xcolor}
\PassOptionsToPackage{quiet}{fontspec}
\documentclass[lang=sv,ptsize=10pt,font=none,nomath,titles=bf]{skrapport}
%% Packages
% Packages included by <skrapport>:
% polyglossia, microtype, icomma, amsmath,
% unicode-math, xunicode, fontspec
% subfiles
\usepackage{subfiles}
% graphics
%\usepackage[usenames,dvipsnames]{xcolor}
\usepackage{graphicx}
\usepackage{tikz}
\usetikzlibrary{shadows}
% page size
\paper{a4}{%
	\usepackage[a4paper,xetex]{geometry}%
	\usepackage[cam,info,a4,center]{crop}%
}
\paper{c5}{%
	\usepackage[c5paper,xetex]{geometry}%
	\usepackage[cam,info,c5,center]{crop}%
}
% code
\usepackage{minted}
% improvement packages
\usepackage{multirow}
\usepackage[swedish]{varioref}
\usepackage[hyperfootnotes=false,colorlinks=false,linkcolor=Red,urlcolor=Magenta]{hyperref}
\usepackage[para,perpage,ragged,bottom,stable]{footmisc}
\usepackage{enumitem}
\usepackage{fancyhdr}
\usepackage{booktabs}
% references
\usepackage[swedish]{chscite}
% other stuff
\usepackage{ifdraft}
\usepackage{siunitx}
\usepackage{hologo}
\let\sups\relax
\usepackage[safe,noenc]{tipa}
\usepackage[titles]{tocloft}
\usepackage{titlesec}
\usepackage[font=small,format=plain,labelfont=bf,textfont=it]{caption}
\usepackage{subfig}
\usepackage{sidecap}
\usepackage{cprotect}
\usepackage{bookmark}
\usepackage{xspace}
\usepackage{mathtools}
% Source code line numbers in margin
%\usepackage{srcline}
% Paket som demonstreras i del 6
\usepackage{braket}
\usepackage{amscd}
\DeclareMathOperator\add{add}
\DeclareMathOperator\cov{cov}
\DeclareMathOperator\non{non}
\DeclareMathOperator\cf{cf}

%% Draft fixes to minted, tikz
\ExplSyntaxOn
\ifdraft{%
	% enable skrapport draft mode
	\bool_gset_true:N\g__skrapport_draft_bool
	\__skrapport_setup_draft:
	% this probably broken.
	\cs_if_exist_use:cT{define@key}{{FV}{mathescape}{}}
	% Replace minted with regular environments
	\RenewDocumentCommand\mint{omv}{%
		\texttt{#3}
	}
	\RenewDocumentEnvironment{minted}{om}{%
		\VerbatimEnvironment%
		\fvset{#1}%
		\begin{Verbatim}%
	}{%
		\end{Verbatim}%
	}
	\RenewDocumentCommand\inputminted{omm}{%
		\fvset{#1}%
		\VerbatimInput{#3}%
	}
	% LaTeX source code
	\newminted{latex}{frame=single}%
	\newminted{text}{frame=single}%
	\newminted{r}{frame=single}%
	\newminted{matlab}{frame=single}%
	\newminted{gnuplot}{frame=single}%
	\newmint{latex}{frame=single}%
	% Line numbers in margin
	%\lnoson% % Disable when not spellchecking/proofreading
}{%
	% Common options
	\cs_set_nopar:Nn\l__tmpa_cs:{
		frame=single,
		bgcolor=mintedbg,
		rulecolor=\color{mintedbg}
	}
	% Define environments
	\exp_args:Nno\newminted{latex}{\l__tmpa_cs:}
	\exp_args:Nno\newminted{text}{\l__tmpa_cs:}
	\exp_args:Nno\newminted{r}{\l__tmpa_cs:}
	\exp_args:Nno\newminted{matlab}{\l__tmpa_cs:}
	\exp_args:Nno\newminted{gnuplot}{\l__tmpa_cs:}
	\exp_args:Nno\newmint{latex}{\l__tmpa_cs:}
	% Undefine the temporary macro
	\cs_undefine:N\l__tmpa_cs:
}
\ExplSyntaxOff

%% FONT SELECTION
\ExplSyntaxOn

%% Set the font directory
\cs_if_exist:NF\latexbokFontdir
	{\cs_gset:Npn\latexbokFontdir{fonts}}

%% Font features
\defaultfontfeatures{
	Numbers={Proportional,OldStyle},
	Ligatures={Historic,Common,Required,Contextual},
	Contextuals=Swash,
	SmallCapsFeatures={
		Letters=SmallCaps,
		Scale=0.96,
		Weight=0.8
	},
}
%% Main font: Alegreya
\setmainfont[Scale=1.05,
		Path=\latexbokFontdir/alegreya/,
		ItalicFont=Alegreya-Italic.ttf,
		BoldFont=Alegreya-Bold.ttf,
		BoldItalicFont=Alegreya-BoldItalic.ttf
	]{Alegreya-Regular.ttf}
%% Sans serif font: Open Sans
\setsansfont[Scale=0.9,
		Path=\latexbokFontdir/open-sans/,
		ItalicFont=OpenSans-Italic.ttf,
		BoldFont=OpenSans-Bold.ttf,
		BoldItalicFont=OpenSans-BoldItalic.ttf
	]{OpenSans-Regular.ttf}
%% Monospace font: Consola mono
\setmonofont[Scale=0.9,
		Path=\latexbokFontdir/consola-mono/,
		BoldFont=ConsolaMono-Bold.ttf,
	]{ConsolaMono-Regular.ttf}
%% Title font: Avería Serif
\newfontfamily\titlefont[Scale=1.05,
		Path=\latexbokFontdir/averia-serif/,
		ItalicFont=AveriaSerif-Italic.ttf,
		BoldFont=AveriaSerif-Bold.ttf,
		BoldItalicFont=AveriaSerif-BoldItalic.ttf
	]{AveriaSerif-Regular.ttf}

%% UTF and TIPA fonts
\cs_set_eq:NN\utffont\rmfamily
\cs_set_eq:NN\tipafont\sffamily

%% Patch the title style to include the new font
\cs_generate_variant:Nn\cs_set_protected:Nn{No}
\cs_set_protected:No\__skrapport_title_style:
	{\__skrapport_title_style:\exp_not:N\titlefont}

\ExplSyntaxOff

%% Hacks
\ExplSyntaxOn
% Appendix
\crefname{appendix}{appendix}{appendix}
\Crefname{appendix}{Appendix}{Appendix}
\apptocmd{\appendix}{%
  \crefalias{part}{appendix}%
}{}{}
% Listings
\DeclareFloatingEnvironment[fileext=lok,placement={tb},name=Exempel]{kod}
\newsubfloat{kod}
\crefname{kod}{exempel}{exempel}
\Crefname{kod}{Exempel}{Exempel}
% Displaying one line of code
\NewDocumentCommand\kodrad{mm}{
  \hspace{1ex}
  \inputminted[firstline=#1,lastline=#1]{latex}{#2.tex}
}
% File last modified
\def\parsedate #1:2#2#3#4#5#6#7#8#9\empty{\ifx{#2}{9}19\else20\fi#3#4/#5#6/#7#8}
\NewDocumentCommand\moddate{o}{
  \exp_args:Nx\parsedate\pdffilemoddate{
    \IfNoValueTF{#1}{\jobname.tex}{#1}
  }\empty
}
% Minted background fix
\makeatletter
\RenewDocumentEnvironment{minted@colorbg}{m}{
	\par
	\def\minted@bgcol{#1}
	\noindent\begin{lrbox}{\minted@bgbox}
	\begin{minipage}{\linewidth-2\fboxsep}
}{
	\end{minipage}
	\end{lrbox}
	\colorbox{\minted@bgcol}{\usebox{\minted@bgbox}}
	\par
}
% Fix for imakeidx
\AtBeginDocument{\def\alsoname{se~även}}
\makeatother
\ExplSyntaxOff

%% Convenient definitions
\ExplSyntaxOn
%% Set the examples directory
\cs_if_exist:NF\latexbokFiguredir
	{\cs_gset:Npn\latexbokFiguredir{examples}}
% PDF format
\NewDocumentCommand\PDF{}
	{\textsc{PDF}\xspace}
% PNG format
\NewDocumentCommand\PNG{}
	{\textsc{PNG}\xspace}
% DVI format
\NewDocumentCommand\DVI{}
	{\textsc{DVI}\xspace}
% EPS format
\NewDocumentCommand\EPS{}
	{\textsc{EPS}\xspace}
% JPEG format
\NewDocumentCommand\JPEG{}
	{\textsc{JPEG}\xspace}
% UTF-8 standard
\NewDocumentCommand\UTF{}
	{\textsc{UTF-8}\xspace}
% Command-line input
\NewDocumentCommand\cli{m}
	{\texttt{#1}}
% CLI option
\NewDocumentCommand\opt{m}
	{\texttt{\emph{#1}}}
% LaTeX package
\NewDocumentCommand\pack{m}
	{\textsf{#1}}
% pdfLaTeX logotype
\NewDocumentCommand\pdfLaTeX{}
	{\hologo{pdfLaTeX}\xspace}
% BibTeX logotype
\NewDocumentCommand\BibTeX{}
	{\hologo{BibTeX}\xspace}
% TikZ logotype
\NewDocumentCommand\TikZ{}
	{Ti\textit{k}Z\xspace}
% PGF/TikZ logotype
\NewDocumentCommand\PGFTikZ{}
	{PGF/\TikZ}
% XeTeX logotype
\NewDocumentCommand\XeTeX{}
	{\hologo{XeTeX}\xspace}
% LyX logotype
\NewDocumentCommand\LyX{}
	{\hologo{LyX}\xspace}
% MATLAB logotype
\NewDocumentCommand\MATLAB{}
	{\textsc{Matlab©}\xspace}
% GNUplot logotype
\NewDocumentCommand\gnuplot{}
	{gnuplot\xspace}
% R logotype
\NewDocumentCommand\Rlogo{}
	{R\xspace}
% Mathematica logotype
\NewDocumentCommand\Mathematica{}
	{Mathematica\xspace}
% english translation
\NewDocumentCommand\eng{m}
	{(eng.~\textenglish{\emph{#1})}}
% LaTeX command
\NewDocumentCommand\cmd{m}
	{\textenglish{\texttt{\textbackslash{}#1}}}
% LaTeX environment
\NewDocumentCommand\env{m}
	{\textenglish{\texttt{#1}}}
% minted style
\usemintedstyle{solarizedlight}
\definecolor{mintedbg}{rgb}{0.9665,0.9550,0.9175}
% Unit definitions
\DeclareSIUnit\point{pt}
\DeclareSIUnit\em{em}
\DeclareSIUnit\mu{mu}
\sisetup{locale=DE}
% Math redefinitions
\DeclareMathOperator{\erfc}{erfc}
\RenewDocumentCommand\frac{mm}{
	\genfrac{}{}{}{}{\displaystyle #1}{\displaystyle #2}
}
\RenewDocumentCommand\d{m}{
	\ensuremath{
		\;\mathrm{d}#1
		\peek_meaning:NTF\d{\!}{}
	}
}
% Dash
\DeclareDocumentCommand\dash{}{
    \cs_if_exist_use:NT\texorpdfstring
    {\unskip\nobreak\thinspace\textemdash\thinspace\ignorespaces}{{ - }}
}
\cs_if_exist_use:NT\DeclareUnicodeCharacter{{2014}{\dash}}
\ExplSyntaxOff
% Useful colors
\definecolor{required}{rgb}{0.384,0.576,0.000}
\definecolor{optional}{rgb}{0.992,0.843,0.000}
\definecolor{unavailable}{rgb}{0.863,0.196,0.184}
\ExplSyntaxOff

%% Redefinitions of style
\ExplSyntaxOn
% Section numbering depth
\setcounter{secnumdepth}{1}
% Table of contents depth
\setcounter{tocdepth}{2}
% Table of contents bf sections
\RenewDocumentCommand\cftsecfont{}{\bfseries}
% Fancy page style
\fancypagestyle{thefancy}{
	\fancyhead{}
	\fancyfoot{}
	\fancyhead[LE,RO]{\thepage}
	\cs_gset_nopar:Npn\headrulewidth{0.1pt}
}
\dim_set:Nn\headheight{15pt}
% Restyle the section command
\RenewDocumentCommand\thesection{}{\Roman{section}}
\cs_generate_variant:Nn\__skrapport_generic_section:nnnnn{nnnon}
\cs_set:Npn\__skrapport_section_style:
  {\normalfont\__skrapport_title_style:}
\dim_set:Nn \c__skrapport_section_indent_dim{.25\textwidth} % magic number
\RenewDocumentCommand\section{som}{
  \cleardoublepage
  \group_begin:
    \centering
    \fontsize{36pt}{36pt}\selectfont
    \__skrapport_generic_section:nnnon{section}{1}{#1}{
      \IfNoValueTF{#2}{#3}{#2}
    }{
      \newline\nobreak{\LARGE #3}
    }
  \group_end:
  \bigskip
}
% Holy shit, hacking my own class
% Undefine paragraph counters
\makeatletter
\cs_undefine:N\c@paragraph
\cs_undefine:N\c@subparagraph
\cs_undefine:N\paragraph
\cs_undefine:N\subparagraph
\makeatother
% Define new subsubsubsection
\skip_const:Nn\c__skrapport_subsubsubsection_pre_skip
  {-2ex~plus~.5ex~minus~-.5ex}
\skip_const:Nn\c__skrapport_subsubsubsection_post_skip
  {.125ex~plus~.125ex}
\dim_const:Nn \c__skrapport_subsubsubsection_indent_dim
  {\c_zero_dim}
\cs_set_protected:Nn\__skrapport_subsubsection_style:
  {\normalfont\large\__skrapport_title_style:\itshape}
\DeclareDocumentCommand\subsubsubsection{som}{
  \__skrapport_generic_section:nnnnn{subsubsubsection}{3}{#1}{#2}{#3}
}
% Redefine TOC entries to match
\newlistentry[subsubsection]{subsubsubsection}{toc}{3}
\newlistentry[subsubsubsection]{paragraph}{toc}{4}
\newlistentry[paragraph]{subparagraph}{toc}{5}
\cftsetindents{subsubsubsection}{6.45em}{4.15em}
\cftsetindents{paragraph}{10.6em}{5.15em}
\cftsetindents{subparagraph}{15.75em}{6.15em}
\cs_set:Npn\toclevel@subsubsubsection{4}
\cs_set:Npn\toclevel@paragraph{5}
\cs_set:Npn\toclevel@subparagraph{6}
% Redefine the counter macros
\cs_set_nopar:Npn\thesubsubsubsection
  {\thesubsubsection.\arabic{subsubsubsection}}
\cs_set_nopar:Npn\theparagraph
  {\thesubsubsubsection.\arabic{paragraph}}
\cs_set_nopar:Npn\thesubparagraph
  {\theparagraph.\arabic{subparagraph}}
\ExplSyntaxOff


%% Load the bibliography data
\addbibresource{referenser.bib}

%% METADATA
\title{Att \hologo{TeX}a: en praktisk guide}
\author{Simon Sigurdhsson}
\regarding{Nybörjarguide i \hologo{LaTeX}, andra upplagan.}
\date{}

%% Actual document
\begin{document}
	%% Title page
	\subfile{chapters/meta/titlepage.tex}
	\cleardoublepage 
	%% Table of contents
	\frontmatter
	\pagestyle{thefancy}
	\begingroup
		\ExplSyntaxOn
		\cs_set_eq:NN\__old_pack\pack
		\cs_set_eq:NN\__old_cmd\cmd
		\cs_set_eq:NN\__old_env\env
		\DeclareDocumentCommand\pack{som}{\textsf{#3}}
		\DeclareDocumentCommand\cmd{som}{\texttt{\textbackslash{}#3}}
		\DeclareDocumentCommand\env{som}{\texttt{#3}}
		\ExplSyntaxOff
		\tableofcontents
		\ExplSyntaxOn
		\cs_set_eq:NN\pack\__old_pack
		\cs_set_eq:NN\cmd\__old_cmd
		\cs_set_eq:NN\env\__old_env
		\ExplSyntaxOff
	\endgroup
	\cleardoublepage
	\mainmatter
	%% INLEDNING
	\subfile{chapters/0/0.tex}
	%% GRUNDLÄGGANDE BEGREPP
	\subfile{chapters/1/1.tex}
	%% TYPSÄTTNING MED LATEX
	\subfile{chapters/2/2.tex}
	%% MATEMATIK MED LATEX OCH AMS
	\subfile{chapters/3/3.tex}
	%% GRAFIK MED LATEX
	\subfile{chapters/4/4.tex}
	%% REFERENSER MED BIBTEX
	\subfile{chapters/5/5.tex}
	%% VIDARE LÄSNING
	\subfile{chapters/6/6.tex}
	%% BIBLIOGRAFI
	\cleardoublepage
	\let\oldurl=\url
	\paper{a4}{\renewcommand\url[1]{\mbox{\small\oldurl{#1}}}}
	\paper{c5}{\renewcommand\url[1]{\mbox{\footnotesize\texttt{\oldurl{#1}}}}}
	\defbibheading{bibliography}[\bibname]{\nntocsection{#1}}
	\printbibliography
	\let\url\oldurl
	%% INDEX
	% korsrefererande index
	% kapitel 0?
	\index{mellanrum|see{tomrum}}
	\index{pdfLaTeX@\pdfLaTeX{}|see{kompilator, \pdfLaTeX{}}}
	\index{LuaLaTeX@\hologo{LuaTeX}|see{kompilator, \hologo{LuaTeX}}}
	\index{XeLaTeX@\hologo{XeTeX}|see{kompilator, \hologo{XeTeX}}}
	\index{CTAN|see{Comprehensive \TeX{} Archive Network, The}}
	% kapitel 1?
	\index{separation!mellanrum|see{tomrum}}
	\index{nyradstecken|see{tomrum, nyradstecken}}
	\index{och-tecken|see{specialtecken}}
	\index{\&!i text|see{specialtecken}}
	\index{\&!i tabell|see{tabell}}
	\index{procenttecken|see{specialtecken}}
	\index{\%!i text|see{specialtecken}}
	\index{\%!i \LaTeX-kod|see{kommentar}}
	\index{dollartecken|see{specialtecken}}
	\index{\$!i text|see{specialtecken}}
	\index{\$!matematikläge|see{matematikläge}}
	\index{frivillig parameter|see{parameter}}
	\index{obligatorisk parameter|see{parameter}}
	\index{klass|see{dokumentklass}}
	\index{pappersstorlek|see{inställningar, standardklass}}
	\index{textstorlek!dokumentnivå|see{inställningar, standardklass}}
	\index{kommando!textstorlek|see{textstorlek, kommando}}
	\index{texdoc@\texttt{texdoc}|see{paket, dokumentation}}
	\index{tlmgr@\texttt{tlmgr}|see{paket, installera}}
	\index{dokumentation!paket|see{paket, dokumentation}}
	\index{installera!paket|see{paket, installera}}
	\index{kompilera|see{\PDF, kompilera till}}
	\index{kompilera!automatiskt|see{\cli{latexmk}}}
	% Kapitel 2
	\index{textstycke|see{styckesindelning}}
	\index{kommando!rubrik|see{rubrik}}
	\index{kapitel|see{rubrik}}
	\index{radbrytning!automatisk|see{avstavning}}
	\index{kursiv|see{textstil, kursiv}}
	\index{fetstil|see{textstil, fetstil}}
	\index{kapitäler|see{textstil, kapitäler}}
	\index{serif-typsnitt|see{typsnitt, serif}}
	\index{sans-serif-typsnitt|see{typsnitt, sans-serif}}
	\index{miljö!flytande objekt|see{flytande objekt}}
	\index{referens!kors-|see{korsreferens}}
	\index{lista!ord-|see{ordlista}}
	\index{sakregister|see{register}}
	\index{förteckning|see{register}}
	\index{förteckning!innehålls-|see{innehållsförteckning}}
	\index{tecken!special-|see{specialtecken}}
	\index{tecken!citat-|see{citattecken}}
	\index{indentering|see{styckesindelning, indrag}}
	\index{indrag|see{styckesindelning, indrag}}
	\index{framsida|see{titelsida}}
	\index{mått!i LaTeX@i \LaTeX{}|see{längd}}
	% Kapitel 3
	\index{mått!i text|see{SI-enhet}}
	\index{enhet|see{SI-enhet}}
	% Kapitel 4
	% Kapitel 5
	% Kapitel 6
	\index{böcker|see{resurser, böcker}}
	\index{resurser!CTAN|see{Comprehensive \TeX{} Archive Network, The}}
	\index{resurser!TeX.SE@\TeX.SE|see{\TeX{} Stack Exchange}}
	\index{TeX.SE@\TeX.SE|see{\TeX{} Stack Exchange}}
	% Skriv förteckningar
	\cleardoublepage
	\indexprologue{\label{sec:idx}
		Detta register innehåller en förteckning över olika teman, termer
		och liknande som diskuteras i boken. Den innehåller inte en
		förteckning över de \LaTeX-paket som diskuteras (en sådan återfinns
		på~\cpageref{sec:idx-pkg}) eller de kommandon och miljöer som nämns
		(en förteckning över dessa finns på~\cpageref{sec:idx-cmd}).
	}
	\printindex
	\cleardoublepage
	\indexprologue{\label{sec:idx-pkg}
		Denna förteckning listar alla de \LaTeX-paket som nämns i boken.
	}
	\printindex[packages]
	\cleardoublepage
	\indexprologue{\label{sec:idx-cmd}
		Denna förteckning listar alla de kommandon och mijöer som 
		diskuteras i boken.
	}
	\printindex[macros]
	%% EN ENKEL MALL
	\cleardoublepage
	\appendix
	\subfile{chapters/meta/appendix.tex}
	\backmatter
\end{document}

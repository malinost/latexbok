\ifdefined\latexbokFontdir\else\def\latexbokFontdir{../../fonts}\fi
\ifdefined\latexbokFiguredir\else\def\latexbokFiguredir{../../examples}\fi
\documentclass[lang=sv,ptsize=10pt,font=none,nomath,titles=bf,../../a4.tex]{subfiles}
\begin{document}
\section{En enkel mall}\label{app:1}
Följande mall är tänkt att användas för att skriva enklare rapporter på
Chalmers Tekniska Högskola. Eftersom \LaTeX-installationen på deras
datorer saknar en del paket (\pack{siunitx} och \pack{chscite}) är dessa
bortkommenterade. Vill man ändå använda dem kan man installera dessa i
sin hemmapp, under \texttt{\textasciitilde/texmf/}%
\footnote{Detta förklaras närmre av \textcite[ss.~89–90]{Oetiker11}%
\hfill}.

Kommentarerna och texten i mallen bör förklara och motivera vad som görs
vilka paket som inkluderas någorlunda. Om något är oklart, referera till
motsvarande del av den här introduktionen. Hela mallen finns dessutom
tillgänglig som en \texttt{.tex}-fil för nedladdning%
\footnote{\url{https://raw.github.com/urdh/latexbok/master/examples/A/mall.tex}}.

\ifdraft{\inputminted[frame=single,firstline=1,lastline=12]{latex}{\latexbokFiguredir/A/mall.tex}}{\inputminted[frame=single,bgcolor=mintedbg,rulecolor=\color{mintedbg},firstline=1,lastline=12]{latex}{\latexbokFiguredir/A/mall.tex}}
\ifdraft{\inputminted[frame=single,firstline=13,lastline=53]{latex}{\latexbokFiguredir/A/mall.tex}}{\inputminted[frame=single,bgcolor=mintedbg,rulecolor=\color{mintedbg},firstline=13,lastline=53]{latex}{\latexbokFiguredir/A/mall.tex}}
\ifdraft{\inputminted[frame=single,firstline=54,lastline=94]{latex}{\latexbokFiguredir/A/mall.tex}}{\inputminted[frame=single,bgcolor=mintedbg,rulecolor=\color{mintedbg},firstline=54,lastline=94]{latex}{\latexbokFiguredir/A/mall.tex}}
\ifdraft{\inputminted[frame=single,firstline=95,lastline=118]{latex}{\latexbokFiguredir/A/mall.tex}}{\inputminted[frame=single,bgcolor=mintedbg,rulecolor=\color{mintedbg},firstline=95,lastline=118]{latex}{\latexbokFiguredir/A/mall.tex}}
\end{document}

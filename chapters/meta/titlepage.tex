\ifdefined\latexbokFontdir\else\def\latexbokFontdir{../../fonts}\fi
\ifdefined\latexbokFiguredir\else\def\latexbokFiguredir{../../examples}\fi
\documentclass[10pt,../../a4.tex]{subfiles}
\begin{document}
\pagestyle{empty}
\pagenumbering{Alph}

%\begin{titlepage} % INTERNAL TITLE PAGE (fold)
\publishers{
	%Senast uppdaterad \moddate{}.\\
	Licenserad under Creative Commons By-Sa 3.0.\\[1ex]
	\large\url{http://creativecommons.org/licenses/by-sa/3.0/}
}
\uppertitleback{
	\begingroup
	\backtrackerfalse
	Den här boken är
	en inkomplett guide till att skriva och typsätta \LaTeX-dokument riktad
	till studenter på Chalmers Tekniska Högskola, specifikt programmen
	Teknisk Matematik och Teknisk Fysik.
	Inspiration har tagits från bland annat \textcites{Schultz05}{Voss10},
	men främst från \textcite{Oetiker11}.
	\endgroup
}

% DEDICATION, THANKS AND PUBLISHER
\lowertitleback{
	Boken är typsatt med \hologo{XeLaTeX}, och designen är baserad på 
	dokumentklassen \textsf{scrbook}, som finns tillgänglig på
	CTAN som en del av KOMA-Script.
	Typsnitten som används är \emph{Linux Libertine}, \emph{Linux Biolinum} och \emph{Bitstream Vera Sans Mono}.
	Dessa typsnitt finns tillgängliga på
	\url{http://www.linuxlibertine.org} och
	\url{https://www-old.gnome.org/fonts/},
	respektive.
	\\[\bigskipamount]
	Källkoden till boken finns tillgänglig på
	\url{http://github.com/urdh/latexbok/}.
	\\[\bigskipamount]
	%Tryck: XXX\\
	Göteborg, 2013
}
\dedication{
%		{\csname__skrapport_title_style:\endcsname\huge Tillägnad}\\[1ex]
%		någon?
%		\bigskip\bigskip\bigskip
	{\usekomafont{disposition}\Large Tack till}\\[1.5ex]
	\emph{Phaddergrupp 255},\\
	som korrekturläst och\\
	hjälpt till att förbättra boken\\[1.5ex]
	{\usekomafont{disposition}\Large och}\\[1.5ex]
	\emph{Christian von Schultz},\\
	\emph{Tobias Oetiker} och \emph{Herbert Voß}\\
	för inspirerande texter på samma tema.
}

% Force titleback even in onepage mode
\makeatletter
\paper{a4}{
	\@twosidetrue
	\maketitle
	\@twosidefalse
}
\paper{c5}{
	\maketitle
}
\makeatother

\end{document}

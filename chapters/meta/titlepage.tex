\def\latexbokFontdir{../../fonts}
\def\latexbokFiguredir{../../examples}
\documentclass[lang=sv,ptsize=10pt,font=none,nomath,titles=bf,../../a4.tex]{subfiles}
\begin{document}
\pagestyle{empty}
\pagenumbering{Alph}

\begin{titlepage} % INTERNAL TITLE PAGE (fold)
	\license{%Senast uppdaterad \moddate{}.\\
	Licenserad under Creative Commons By-Sa 3.0.\\
	\url{http://creativecommons.org/licenses/by-sa/3.0/}}
	\maketitle
	\begin{abstract}
		En inkomplett guide till att skriva och typsätta \LaTeX-dokument riktad
		till studenter på Chalmers Tekniska Högskola, specifikt programmen
		Teknisk Matematik och Teknisk Fysik.
		Inspiration har tagits från bland annat \citeasnoun{Schultz05} och
		\citeasnoun{Voss10}, men främst från \citeasnoun{Oetiker11}.
	\end{abstract}
\end{titlepage} % (end)
\cleardoublepage

\begin{center} % DEDICATION, THANKS AND PUBLISHER
	\large\vspace*{36pt}
%		{\makeatletter\@titstyle\makeatother\huge Tillägnad}\\[1ex]
%		någon?
%		\bigskip\bigskip\bigskip

	{\csname__skrapport_title_style:\endcsname\Large Tack till}\\[1ex]
	\emph{Phaddergrupp 255},\\
	som korrekturläst och\\
	hjälpt till att förbättra boken\\[1ex]
	{\csname__skrapport_title_style:\endcsname\Large och}\\[1ex]
	\emph{Christian von Schultz},\\
	\emph{Tobias Oetiker} och \emph{Herbert Voß}\\
	för inspirerande texter på samma tema.
	\vfill 
	
	\small
	%Tryck: XXX\\
	Göteborg, 2012
\end{center}
\end{document}

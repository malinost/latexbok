% arara: pdflatex
% arara: pdflatex
% --------------------------------------------------------------------------
% the LIBERTINEHOLOGOPATCH package
% 
%   patch kerning of hologo's logos for usage with the `libertine' package
% 
% --------------------------------------------------------------------------
% Clemens Niederberger
% Web:    https://bitbucket.org/cgnieder/libertinehologopatch/
% E-Mail: contact@mychemistry.eu
% --------------------------------------------------------------------------
% Copyright 2013 Clemens Niederberger
% 
% This work may be distributed and/or modified under the
% conditions of the LaTeX Project Public License, either version 1.3
% of this license or (at your option) any later version.
% The latest version of this license is in
%   http://www.latex-project.org/lppl.txt
% and version 1.3 or later is part of all distributions of LaTeX
% version 2005/12/01 or later.
% 
% This work has the LPPL maintenance status `maintained'.
% 
% The Current Maintainer of this work is Clemens Niederberger.
% --------------------------------------------------------------------------
% The libertinehologopatch package consists of the files
%  - libertinehologopatch.sty
%  - libertinehologopatch_en.tex, libertinehologopatch_en.pdf
%  - README
% --------------------------------------------------------------------------
% If you have any ideas, questions, suggestions or bugs to report, please
% feel free to contact me.
% --------------------------------------------------------------------------
\PassOptionsToPackage{supstfm=libertinesups}{superiors}
\documentclass{cnpkgdoc}
\docsetup{
  pkg      = libertinehologopatch ,
  cmd      = \lbthlpt ,
  title    = {Libertine Hologo \\ Patch} ,
  subtitle = Patching the logos provided by the \paketfont{hologo}
    package when used together with the \paketfont{libertine} package
    and \pdfLaTeX. ,
  label    =
}

\TitlePicture{%
  \centering \Huge From {\restorelogos\LaTeXTeX} to \LaTeXTeX}

\addcmds{LaTeXTeX,restorelogos,XeLaTeX}

\renewcommand*\othersectionlevelsformat[3]{%
  \textcolor{main}{#3\autodot}\enskip}
\renewcommand*\partformat{%
  \textcolor{main}{\partname~\thepart\autodot}}
\usepackage{fnpct}

\usepackage{filecontents}
\begin{filecontents}{\jobname.ist}
 heading_prefix "{\\bfseries "
 heading_suffix "\\hfil}\\nopagebreak\n"
 headings_flag  1
 delim_0 "\\dotfill\\hyperpage{"
 delim_1 "\\dotfill\\hyperpage{"
 delim_2 "\\dotfill\\hyperpage{"
 delim_r "}\\textendash\\hyperpage{"
 delim_t "}"
 suffix_2p "\\nohyperpage{\\,f.}"
 suffix_3p "\\nohyperpage{\\,ff.}"
\end{filecontents}

\usepackage{imakeidx}
\indexsetup{noclearpage,othercode=\footnotesize}
\makeindex[columns=3,intoc,options={-sl \jobname.ist}]

\begin{document}

\section{License and Requirements}\secidx{License}
\lbthlpt is placed under the terms of the \LaTeX{} Project Public License,
version 1.3 or later (\url{http://www.latex-project.org/lppl.txt}).  It has
the status ``maintained.''

\lbthlpt loads and needs the packages \paket{etoolbox}, \paket{hologo},
\paket{ifxetex}, \paket{ifluatex} and \paket{libertine}.
\secidx*{License}

\section{Options}\secidx{Options}
There are a few package options:
\begin{beschreibung}
 \Option{verbose}\newline
   Inform when patching is successful and warn if it fails. This is the default
   behaviour.
 \Option{silent}\newline
   Neither inform nor warn if patching succeeds or fails.
 \Option{errors}\newline
   Issue an error instead of a warning if patching fails.
 \Option{shortcuts}\newline
   (Re-) Define shortcuts for common logos to use the newly patched \paket{hologo}
   logos: \cmd{TeX} \TeX, \cmd{LaTeX} \LaTeX, \cmd{LaTeXe} \LaTeXe, \cmd{LaTeXTeX}
   \LaTeXTeX, \cmd{pdfTeX} \pdfTeX, \cmd{pdfLaTeX} \pdfLaTeX, \cmd{XeTeX} \XeTeX,
   \cmd{XeLaTeX} \XeLaTeX, \cmd{LuaTeX} \LuaTeX, \cmd{LuaLaTeX} \LuaLaTeX. This
   is default behaviour.
 \Option{noshortcuts}\newline
   Don't (re-) define above mentioned logos.
\end{beschreibung}
\secidx*{Options}

\section{Usage}\secidx{Usage}
The \paket{hologo} package by Heiko Oberdiek provides a comprehensive list of and
access to the most common \TeX-related logos. However, the kerning values it uses
are optimized for use with \TeX's standard font, Computer Modern. I happen to
like the Linux Libertine O and Linux Biolinum O fonts very much. They are used
in this documentation as well. Unfortunately many of the logos look bad with these
fonts to say the least:

\begin{beispiel}
 \restorelogos\huge
 \LaTeX, \XeLaTeX, \LaTeXTeX \par
 \textit{\LaTeX, \XeLaTeX, \LaTeXTeX}
 
 \sffamily
 \LaTeX, \XeLaTeX, \LaTeXTeX \par
 \textit{\LaTeX, \XeLaTeX, \LaTeXTeX}
\end{beispiel}

What \lbthlpt does is it checks if one of these fonts is used and changes the
kerning then:

\begin{beispiel}
 \huge
 \LaTeX, \XeLaTeX, \LaTeXTeX \par
 \textit{\LaTeX, \XeLaTeX, \LaTeXTeX}
 
 \sffamily
 \LaTeX, \XeLaTeX, \LaTeXTeX \par
 \textit{\LaTeX, \XeLaTeX, \LaTeXTeX}
\end{beispiel}

All you have to do is load the package in the usual \LaTeXe{} way:
\begin{beispiel}[code only]
 \usepackage{libertinehologopatch}
\end{beispiel}
and all the logos will be patched appropriately.

\lbthlpt defines additional user macros:
\begin{beschreibung}
 \Befehl{restorelogos}\newline
   Restores the original definitions of the logo commands, the scope is local.
 \Befehl{originallogo}{<logo>}\newline
   This command takes the same argument as \cmd{hologo} and outputs the unpatched
   version of the respective logo. It uses \cmd{restorelogos} internally.
 \Befehl{Originallogo}{<logo>}\newline
   This command takes the same argument as \cmd{Hologo} and outputs the unpatched
   version of the respective uppercased logo. It uses \cmd{restorelogos} internally.
\end{beschreibung}
\secidx*{Usage}

\section{Implementation}
\implementation

\printindex

\end{document}

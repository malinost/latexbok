\documentclass[11pt]{skrapport}
\usepackage[utf8]{inputenc}
\usepackage{harvard}
\usepackage{moreverb}

\newcommand\PDF{\textsc{pdf}}
\newcommand\DVI{\textsc{dvi}}
\newcommand\EPS{\textsc{eps}}
\newcommand\UTF{\textsc{utf8}}
\newcommand\cli[1]{\texttt{#1}}
\newcommand\pack[1]{\textsf{#1}}
\newcommand\pdfLaTeX{pdf\LaTeX}
\newcommand\BibTeX{\textsc{Bib}\TeX}

\begin{document}
	\begin{titlepage} % (fold)
		\title{Att \TeX{}a: en praktisk guide}
		\author{Simon Sigurdhsson\thanks{\url{ssimon@student.chalmers.se}}}
		\regarding{Nybörjarguide i \LaTeX}
		\date{v0.1 (2011--05--22)}
		\maketitle
		\begin{abstract}
			En inkomplett guide till att skriva och typsätta \LaTeX-dokument riktad
			till studenter på Chalmers Tekniska Högskola, specifikt programmen
			Teknisk Matematik och till en viss utsträckning Teknisk Matematik.
			Inspiration har tagits från bl.a. \citeasnoun{Schultz05},
			\citeasnoun{Voss10} och \citeasnoun{Oetiker11}.
		\end{abstract}
	\end{titlepage} % (end)
	
	\section{\LaTeX --- en typsättare}
	\subsection{Vad är \LaTeX?}
	
	\subsection{Varför \LaTeX?}
	Traditionellt så har \LaTeX{} använts för att det gör typsättning av just
	matematik och fysik både enkel och attraktiv. Detta betyder dock inte att
	\LaTeX{} är meningslöst för de som inte behöver denna funktionalitet; långt
	ifrån, till och med. Tillhörande program och paket som till exempel
	\BibTeX{} ger dig som författare frihet att kunna koncentrera dig på att
	skriva, och lämnar över typsättningen och referenshanteringen till
	programmet.
	
	Detta gör att du utan ansträngning kan skapa attraktiva och läsbara
	rapporter, artiklar och böcker.
	
	\section{Ett första \LaTeX-dokument}
	
	\subsection{Konstruktioner}
	
	\subsubsection{Dokumentklasser}
	
	\subsubsection{Paket}
	
	\subsubsection{Miljöer}
	
	\subsubsection{Kommandon}
	
	\subsection{Dokumentstruktur}
	
	\subsubsection{Stycken, avsnitt, m.m.}
	
	\subsection{Ett exempeldokument}
	
	\subsection{En \PDF{} med \pdfLaTeX}
	
% 	\section{Rekommenderade paket}
% 	\subsection{\pack{inputenc} och \pack{fontenc}}
% 	Dessa paket är i princip ett måste när du skriver i \LaTeX. De berättar för
% 	\LaTeX{} vilket format din indata är i, och vilket format dina typsnitt ska
% 	vara i. Generellt sett så kan man förutsätta att indatan kommer att vara
% 	i \UTF-format, och att du vill använda T1-typsnitt; detta görs enkelt:
% 
% \begin{listing}[999]{-2}
% \usepackage[utf8]{inputenc}
% \usepackage[T1]{fontenc}
% \end{listing}
% 
% 	\subsection{\pack{babel}}
% 	Detta paket ger dig avstavning och standardrubriker på rätt språk, och är
% 	ännu ett måste när du skriver rapporter med \LaTeX. Paketet har stöd för en
% 	stor mängd språk, men oftast är det svenska man vill ha:
% 
% \begin{listing}[999]{-1}
% \usepackage[swedish]{babel}
% \end{listing}
% 
% 	\subsection{\AmS\LaTeX --- \pack{amsmath} och \pack{amssymb}}
% 	\AmS, \emph{the American Mathematical Society}, har skapat ett paket som ger
% 	tillgång till en stor mängd \LaTeX-kommandon som underlättar typsättning av
% 	matematik. Det relevanta paketet är egentligen \pack{amsmath}, men ibland
% 	vill man även använda \pack{amssymb} som ger tillgång till en del
% 	matematiska symboler. Paketen inkluderas enkelt:
% 
% \begin{listing}[999]{-2}
% \usepackage{amsmath}
% \usepackage{amssymb}
% \end{listing}
% 	
% 	En stor genomgång av vad som erbjuds av \pack{amsmath} och varför man bör
% 	använda detta paket istället för det som redan finns i \LaTeX{} ges av
% 	\citeasnoun{Voss10}.
% 	
% 	\subsection{Grafik: \pack{graphicx}}
% 	
% 	\subsection{Tabeller: \pack{booktabs}}
% 	
% 	\subsection{Figurtexter: \pack{caption}}
% 	
% 	\section{Skapa \PDF-filer med \pdfLaTeX}
% 	När ditt \LaTeX-dokument förr eller senare ska publiceras på internet,
% 	skrivas ut eller lämnas in till en examinator är det lämpligt att använda
% 	ett utbrett format. \PDF-formatet är ett av de mest utbredda
% 	dokumentformaten och kan öppnas av i princip alla, samtidigt som det enkelt
% 	kan skrivas ut på de flesta datorer. Förr exporterades \LaTeX-dokument till
% 	\PDF-filer genom att konvertera de \DVI-filer \LaTeX{} normalt genererar
% 	till \PDF-filer med hjälp av kommandot \cli{dvipdf} \cite{Schultz05}.
% 	
% 	Numera är detta dock meningslöst; den moderna \LaTeX-varianten \pdfLaTeX{}
% 	kan kompilerar dina \LaTeX-dokument direkt till \PDF-filer, och med paket
% 	som t.ex. \pack{hyperref} kan mycket av funktionaliteten som finns i \PDF{}
% 	användas från \LaTeX.
% 	
% 	Ett problem är dock att typsnitten som normalt används i \LaTeX{} inte är
% 	gjorda för visning på en skärm; de är bitmappstypsnitt och är gjorda för att
% 	se bra ut på papper. Man måste därför använda de T1-typsnitt; dessa är
% 	i vektorformat och fungerar både i utskrifter och på skärmen. Detta gör man
% 	genom att ladda \pack{fontenc}-paketet:
% 	
% \begin{listing}[999]{-1}
% \usepackage[T1]{fontenc}
% \end{listing}
% 
% 	Tyvärr är det gamla standardtypsnittet som används i \LaTeX mycket utdaterat
% 	och saknar många tecken, däribland svenska. Detta innebär att \PDF-filerna
% 	\pdfLaTeX{} genererar inte innehåller ''riktiga'' svenska tecken, och att
% 	man därför inte kan kopiera texten i dem på ett enkelt sätt. Det lättaste
% 	sättet att åtgärda detta är att använda ett annat typsnitt, till exempel
% 	Latin Modern, en modern klon av standardtypsnittet:
% 	
% \begin{listing}[999]{-1}
% \usepackage{lmodern}
% \end{listing}
% 
% 	Det finns även andra typsnitt så som \pack{times} (Times New Roman),
% 	\pack{mathpazo} (Palatino) och en hel del \TeX{} Gyre-typsnitt\footnote{\url{http://www.gust.org.pl/projects/e-foundry/tex-gyre/nfp11.pdf}}.
% 	
% 	En annan aspekt av att använda \pdfLaTeX{} är att det inte stöder import av
% 	\EPS-filer när man använder \pack{graphicx}. De format som istället är
% 	tillgängliga är JPEG, PNG och \PDF{} --- existerande \EPS-filer kan man
% 	konvertera till \PDF{} med hjälp av kommandot \cli{epstopdf}.
% 	
% 	\section{Typografitips}
% 	\subsection{Generella tips för ekvationer}
% 	Funktioner, integraler, \verb|\limits|, vektorer
% 	
% 	\subsection{Siffror och tal}
% 	Decimalavskiljare, tusentalsavskiljare, minustecken
% 	
% 	\subsection{Svensk och utländsk typografi}
% 	tankstreck osv., datum
% 	
% 	\subsection{Fler typografigrejer?}
% 	
% 	\section{Figurer och tabeller}
% 	Floats, varför \LaTeX{} flyttar runt figurer, varför man inte ska
% 	ändra på det, etc.
% 	
% 	\section{Referenssystemet \BibTeX}
	
	\bibliographystyle{dcumod}
	\bibliography{referenser}
\end{document}